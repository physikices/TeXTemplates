%%%% Define Article %%%%%%%%%%%%%%%
\documentclass[
  a4paper,
	12pt
]{article}
%%%%%%%%%%%%%%%%%%%%%%%%%%%%%%%%%%%

%%%% Using Packages %%%%%%%%%%%%%%%
\usepackage{geometry}
\usepackage{graphicx}
\usepackage{amssymb}
\usepackage{amsmath}
\usepackage{amsthm}
\usepackage{empheq}
\usepackage{mdframed}
\usepackage{booktabs}
\usepackage{lipsum}
\usepackage{graphicx}
\usepackage{color}
\usepackage{psfrag}
\usepackage{pgfplots}
\usepackage{bm}
\usepackage[brazil]{babel}
\usepackage[T1]{fontenc}
\usepackage{lmodern}
\usepackage{breqn}
\usepackage{physics}
\usepackage{enumitem}
\usepackage{enumerate}
\usepackage{hyperref}

% Cores Udesc
\usepackage{xcolor}
\definecolor{greenudesc}{RGB}{20,155,85}
\definecolor{greenudescdark}{RGB}{5,75,40}
\definecolor{redudesc}{RGB}{240,65,55}
\definecolor{gray50udesc}{RGB}{128,128,128}
\definecolor{gray65udesc}{RGB}{89,89,89}
\definecolor{gray80udesc}{RGB}{51,51,51}

\definecolor{background}{HTML}{282A36}
\definecolor{currentline}{HTML}{44475A}
\definecolor{foreground}{HTML}{F8F8F2}
\definecolor{comment}{HTML}{6272A4}
\definecolor{dcyan}{HTML}{8BE9FD}
\definecolor{dgreen}{HTML}{50FA7B}
\definecolor{dpink}{HTML}{FF79C6}
\definecolor{dpurple}{HTML}{BD93F9}
\definecolor{dred}{HTML}{FF5555}
\definecolor{dyellow}{HTML}{F1FA8C}
\definecolor{dorange}{HTML}{FFB86C}

%%%%%%%%%%%%%%%%%%%%%%%%%% Define an orangebox command %%%%%%%%%%%%%%%%%%%%%%%%
\newcommand\orangebox[1]{\fcolorbox{ocre}{mygray}{\hspace{1em}#1\hspace{1em}}}
%%%%%%%%%%%%%%%%%%%%%%%%%%%%%%%%%%%%%%%%%%%%%%%%%%%%%%%%%%%%%%%%%%%%%%%%%%%%%%%
\newcommand*\circled[1]{\tikz[baseline=(char.base)]{
            \node[shape=circle,draw,inner sep=2pt] (char) {#1};}}
% overbar and overline bar
\newcommand{\overbar}[1]{\mkern 1.5mu\overline{\mkern-1.5mu#1\mkern-1.5mu}\mkern 1.5mu}
%%%%%%%%%%%%%%%%%%%%%%%%%% Define some useful commands %%%%%%%%%%%%%%%%%%%%%%%%
\newcommand\mx[1]{\begin{math}#1\end{math}}% math expression
\newcommand\mb[1]{\mathbb{#1}}% real numbers
\DeclareMathOperator{\sen}{sen}
\DeclareMathOperator{\senh}{senh}
\DeclareMathOperator{\tg}{tg}
\DeclareMathOperator{\tgh}{tgh}
\DeclareMathOperator{\diag}{diag}
%%%%%%%%%%%%%%%%%%%%%%%%%%%%%%%%%%%%%%%%%%%%%%%%%%%%%%%%%%%%%%%%%%%%%%%%%%%%%%%


%%%%%%%%%%%%%%%%%%%%%%%%%%%% English Environments %%%%%%%%%%%%%%%%%%%%%%%%%%%%%
\newtheoremstyle{mytheoremstyle}{3pt}{3pt}{\normalfont}{0cm}{\rmfamily\bfseries}{}{1em}{{\color{black}\thmname{#1}~\thmnumber{#2}}\thmnote{\,--\,#3}}
\newtheoremstyle{myproblemstyle}{3pt}{3pt}{\normalfont}{0cm}{\rmfamily\bfseries}{}{1em}{{\color{black}\thmname{#1}~\thmnumber{#2}}\thmnote{\,--\,#3}}
\theoremstyle{mytheoremstyle}
\newmdtheoremenv[linewidth=1pt,backgroundcolor=shallowGreen,linecolor=deepGreen,leftmargin=0pt,innerleftmargin=20pt,innerrightmargin=20pt,]{theorem}{Teorema}[section]
\theoremstyle{mytheoremstyle}
\newmdtheoremenv[linewidth=1pt,backgroundcolor=shallowBlue,linecolor=deepBlue,leftmargin=0pt,innerleftmargin=20pt,innerrightmargin=20pt,]{definition}{Definição}[section]
\theoremstyle{myproblemstyle}
\newmdtheoremenv[linecolor=black,leftmargin=0pt,innerleftmargin=10pt,innerrightmargin=10pt,]{problem}{Problema}[section]
%%%%%%%%%%%%%%%%%%%%%%%%%%%%%%%%%%%%%%%%%%%%%%%%%%%%%%%%%%%%%%%%%%%%%%%%%%%%%%%

%%%%%%%%%%%%%%%%%%%%%%%%%%%%%%%%%%%

% Other Settings

%%%% Plotting Settings %%%%%%%%%%%%
\usepgfplotslibrary{colorbrewer}
\pgfplotsset{width=8cm,compat=1.9}
%%%%%%%%%%%%%%%%%%%%%%%%%%%%%%%%%%%

%%%% Title & Author %%%%%%%%%%%%%%%
\title{Lista de Exercícios}
\author{Rodrigo Nascimento}
\hypersetup{%
	pdftitle={\@title},%
	pdfauthor={\@author},%
	colorlinks=true,
	linkcolor=ocre,
	citecolor=deepBlue,
	urlcolor=deepGreen,
	bookmarksdepth=4%
}

%%%%%%%%%%%%%%%%%%%%%%%%%%%%%%%%%%%

\begin{document}
\maketitle
\begin{center}
	\noindent\textcolor{gray}{\rule{\textwidth}{0.5pt}}
\end{center}


\begin{problem}
	Mostre que $\va{v}=\vu{e}_{x}$ e $\va{w}=\vu{e}_{y}$ são vetores lineramente independentes. Calcule também a norma dos dois vetores.	
\end{problem}
\textcolor{deepGreen}{\textbf{Solução:}}

$\va{v}$ e $\va{w}$ serão linearmente independentes se
\begin{dmath}
	\alpha \va{v} + \beta \va{w} = 0\condition{com $\alpha, \beta \in \mathbb{R}$}
\end{dmath}
uma vez que
\begin{align*}
	\alpha\va{v} &= \alpha \vu{e}_{x} = \alpha
	\begin{pmatrix}
		1\\0	
	\end{pmatrix}, &
	\beta\va{w} &= \beta \vu{e}_{y} = \beta
	\begin{pmatrix}
		0\\1	
	\end{pmatrix}
\end{align*}
tem-se que
\begin{align*}
	\alpha \va{v}+\beta \va{w} &= \alpha	
	\begin{pmatrix}
		1\\0	
	\end{pmatrix} + \beta
	\begin{pmatrix}
		0\\1	
	\end{pmatrix} = 
	\begin{pmatrix}
		0\\0	
	\end{pmatrix}\condition{se e somente se $\alpha,\beta=0$}
\end{align*}
logo
\begin{dseries}[frame]
	\mx{\va{v}=\vu{e}_{x}},	
	\mx{\va{w}=\vu{e}_{y}}\condition{são linearmente independentes}
\end{dseries}

A norma (ou módulo) destes dois vetores é dada por
\begin{dgroup*}
	\begin{dmath*}
		\norm{\va{v}} = \sqrt{\vu{e}_{x}\cdot\vu{e}_{x}}=\delta_{x x}=1
	\end{dmath*}
	\begin{dmath*}
		\norm{\va{w}} = \sqrt{\vu{e}_{y}\cdot\vu{e}_{y}}=\delta_{y y}=1	
	\end{dmath*}
\end{dgroup*}

% --------------------------------------------- %
\begin{center}
	\noindent\textcolor{mygray}{\rule{\textwidth}{0.5pt}}
\end{center}
% --------------------------------------------- %
\begin{problem}
	Considere os seguintes vetores no $\mathbb{R}^{2}: \va{v}=\qty(1,2)^{T}\qq{e}\va{w}=\qty(-1,1)^{T}$.
	\begin{itemize}
		\item[a.] Estes vetores são linearmente independentes?
		\item[b.] Escreva qualquer vetor $\va{x}=\qty(x_{1},x_{2})$ na base dada por $\va{v}\qq{e}\va{w}$.
	\end{itemize}
\end{problem}
\textcolor{deepGreen}{\textbf{Solução:}}
\begin{itemize}
	\item[a.] Desde que
		\begin{align*}
			\alpha \va{v}+\beta \va{w} &= 0\\
			\alpha\mqty(1\\2)+\beta\mqty(-1\\1) &= 0
		\end{align*}
		isto é
		\begin{equation*}
			\begin{cases}
				\alpha-\beta = 0\\
				2 \alpha+\beta =0
			\end{cases}	\implies \alpha,\beta=0\qc\forall \alpha,\beta \in \mathbb{R} 
		\end{equation*}
		conclui-se que $\boxed{\va{v}\qq{e}\va{w}\qq{são linearmente independentes}}$
	\item[b.] Escolhendo arbitrariamente $\alpha=4\qq{e} \beta=6$, então $\va{x}=\qty(x_{1},x_{2})$ é dado por
		\begin{align*}
			\mqty(x_{1}\\ x_{2}) &= 4\mqty(1\\2)+6\mqty(-1\\1)
		\end{align*}
		logo $\va{x}=\qty(-2,8)^{T}$
\end{itemize}
% --------------------------------------------- %
\begin{center}
	\noindent\textcolor{mygray}{\rule{\textwidth}{0.5pt}}
\end{center}
% --------------------------------------------- %
\begin{problem}
	Determinar todos os vetores do $\R^{3}$ ortogonais ao vetor $\va{v}=\qty(2,0,1)^{T}$.
\end{problem}
\textcolor{deepGreen}{\textbf{Solução:}}
\begin{definition}
	Dois vetores quaisquer $|{u}\rangle \qq{e} |{v}\rangle$ são ortogonais se
	\begin{align}
		\braket{u}{v} &= 0	
	\end{align}
\end{definition}
Definindo $\langle{u}|=\qty(u_{1},u_{2},u_{3})$ tem-se
\begin{align*}
	\braket{u}{v} &= \qty(u_{1},u_{2},u_{3})\cdot\mqty(2\\0\\1)\\
								&= 2u_{1}+u_{3}=0\implies u_{1}=-\frac{u_{3}}{2}
\end{align*}
logo, qualquer vetor da forma $\langle{u}|=\qty(-u_{3}/2,u_{2},u_{3})$ será ortogonal à $|{v}\rangle$
% --------------------------------------------- %
\begin{center}
	\noindent\textcolor{mygray}{\rule{\textwidth}{0.5pt}}
\end{center}
% --------------------------------------------- %
\begin{problem}
	Mostrar que vale a identidade $\qty(\va{u}\cp \va{v})\cp \va{w}=\qty(\va{w}^{T}\cdot \va{u})\cdot \va{v}-\qty(\va{w}^{T}\cdot \va{v})\cdot \va{u}$. Deduzir que $\va{u}^{T}\cdot\qty(\va{u}\cp \va{v})$	
\end{problem}
Considere a definição a seguir
\begin{definition}{\label{def:prod_vetorial}}
	O produto vetorial entre dois vetores em termos do símbolo de Levi-Civita é	
	\begin{align}
		\va{u}\cp \va{v} &= \sum_{i,j,k} \varepsilon_{ijk}u_{i}v_{j}\vu{e}_{k}
	\end{align}
\end{definition}
deseja-se demostrar que
\begin{align*}
	\qty(\va{u}\cp \va{v})\cp \va{w} &= \qty(\va{w}^{T}\cdot \va{u})\cdot \va{v}-\qty(\va{w}^{T}\cdot \va{v})\cdot \va{u}
\end{align*}
\begin{proof}
	Utilizando a definição \ref{def:prod_vetorial}, tem-se que
	\begin{align*}
		\qty(\va{u}\cp\va{v})\cp\va{w} &= \sum_{i,j,k} \varepsilon_{ijk}\qty(\va{u}\cp\va{v})_{i}w_{j}\vu{e}_{k}
	\end{align*}
	dado que
	\begin{align*}
		\boxed{
			\qty(\va{u}\cp\va{v})_{i} = \sum_{j,k} \varepsilon_{ijk}u_{j}v_{k}
		}
	\end{align*}
	ficamos com
	\begin{align*}
		\qty(\va{u}\cp\va{v})\cp\va{w} &= \sum_{i,j,k} \varepsilon_{ijk}\qty(\sum_{l,m} \varepsilon_{lmi}\va{u}_{l}\va{v}_{m})\va{w}_{j}\vu{e}_{k}\\
																	 &=\sum_{i,j,k,l,m} \varepsilon_{ijk} \varepsilon_{lmi}\va{u}_{l}\va{v}_{m}\va{w}_{j}\vu{e}_{k}
	\end{align*}
	usando a relação
	\begin{align*}
		\boxed{
			\sum_{i}\varepsilon_{ijk}\varepsilon_{lmi} = \delta_{jl}\delta_{km}-\delta_{jm}\delta_{kl}	
		}
	\end{align*}
	tem-se
	\begin{align*}
		\qty(\va{u}\cp\va{v})\cp\va{w} &= \sum_{j,k,l,m} \qty(\delta_{jl}\delta_{km}-\delta_{jm}\delta_{kl})\va{u}_{l}\va{v}_{m}\va{w}_{j}\vu{e}_{k}\\
																	 &= \sum_{j,k,l,m} \delta_{jl} \delta_{km}\va{u}_{l}\va{v}_{m}\va{w}_{j}\vu{e}_{k}-\sum_{j,k,l,m} \delta_{jm} \delta_{kl}\va{u}_{l}\va{v}_{m}\va{w}_{j}\vu{e}_{k}\\
																	 &= \sum_{j,l}\va{w}_{j}\va{u}_{l} \delta_{jl}\sum_{k,m}\vu{e}_{k}\va{v}_{m} \delta_{km}-\sum_{j,m}\va{w}_{j}\va{v}_{m} \delta_{jm}\sum_{k,l}\vu{e}_{k}\va{u}_{l}
	\end{align*}
	usando a definição
	\begin{definition}{\label{def:prod_escalar}}
		Dado dois vetores $\va{u}\qq{e}\va{v}$, o produto escalar entre estes vetores é, por definição
		\begin{align}
			\va{u}^{T}\cdot\va{v} = \sum_{i,j}u_{i}v_{j} \delta_{ij}
		\end{align}
	\end{definition}
	logo
	\begin{align*}
		\qty(\va{u}\cp\va{v})\cp\va{w} &= \qty(\va{w}^{T}\cdot\va{u})\va{v}-\qty(\va{w}^{T}\cdot\va{v})\va{u}
	\end{align*}
\end{proof}

A segunda demonstração envolve o produto misto
\begin{align*}
	\va{u}^{T}\cdot\qty(\va{u}\cp\va{v}) &= 0
\end{align*}
\begin{proof}
	Usando a definição \ref{def:prod_escalar}, devemos ter
	\begin{align*}
		\va{u}^{T}\cdot\qty(\va{u}\cp\va{v}) &= \sum_{i,j}u_{i}\qty(\va{u}\cp\va{v})_{j} \delta_{ij}\\
																				 &= \sum_{i,j}u_{i}\qty(\sum_{i,k} \varepsilon_{ijk}u_{i}v_{k}) \delta_{ij}\\
																				 &= \sum_{i,j,k}\varepsilon_{ijk}u_{i}^{2}v_{k} \delta_{ij}\\
																				 &= \sum_{i,j,k}\varepsilon_{ijk} \delta_{ij}u_{i}^{2}v_{k}\\
																				 &= 0  
	\end{align*}

\end{proof}
% --------------------------------------------- %
\begin{center}
	\noindent\textcolor{mygray}{\rule{\textwidth}{0.5pt}}
\end{center}
% --------------------------------------------- %

\begin{problem}
	Usando a definição de produto vetorial em termos dos versores de base, deduzir as propriedades do produto vetorial de dois vetores qualquer.	
\end{problem}

% --------------------------------------------- %
\begin{center}
	\noindent\textcolor{mygray}{\rule{\textwidth}{0.5pt}}
\end{center}
% --------------------------------------------- %

\end{document}

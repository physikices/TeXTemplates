\setlength{\absparsep}{18pt} % ajusta o espaçamento dos parágrafos do resumo
\begin{resumo}
 O resumo deve ressaltar o
 objetivo, o método, os resultados e as conclusões do documento. A ordem e a extensão
 destes itens dependem do tipo de resumo (informativo ou indicativo) e do
 tratamento que cada item recebe no documento original. O resumo deve ser
 precedido da referência do documento, com exceção do resumo inserido no
 próprio documento. O resumo deve ser escrito como um parágrafo único, sem utilizar referências bibliográficas e evitando ao máximo, o uso de siglas/abreviações. O resumo deve conter até X palavras, sendo composto das seguintes partes (organização lógica): introdução, objetivos, justificativa, metodologia e resultados esperados. Esta é a sequência lógica, não devendo ser utilizados títulos e subtítulos. Não abuse na contextualização, pois o foco deve ser nos objetivos e nos resultados esperados. (\ldots) As palavras-chave devem figurar logo abaixo do
 resumo, antecedidas da expressão Palavras-chave:, separadas entre si por
 ponto e finalizadas também por ponto.

\textbf{Palavras-chave}: \firstkey ; \secondkey ; \thirdkey.
\end{resumo}
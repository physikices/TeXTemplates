%-----------------------------------------------%
% Modelo de Plano de Aula com três momentos pedagógicos
%
% Autor: Rodrigo Nascimento (2022-08-12)
%-----------------------------------------------%

\documentclass[
	% -- opções da classe memoir --
	12pt,				% tamanho da fonte
	openright,			% capítulos começam em pág ímpar (insere página vazia caso preciso)
	oneside,			% twoside para impressão em verso e anverso. Oposto a oneside
	a4paper,			% tamanho do papel. 
	% -- opções da classe abntex2 --
	chapter=TITLE,		% títulos de capítulos convertidos em letras maiúsculas
	%section=TITLE,		% títulos de seções convertidos em letras maiúsculas
	%subsection=TITLE,	% títulos de subseções convertidos em letras maiúsculas
	%subsubsection=TITLE,% títulos de subsubseções convertidos em letras maiúsculas
	% -- opções do pacote babel --
	english,			% idioma adicional para hifenização
%	french,				% idioma adicional para hifenização
%	spanish,			% idioma adicional para hifenização
	brazil				% o último idioma é o principal do documento
]{abntex2}
\selectlanguage{brazil}
%-----------------------------------------------%
% Informações do DOCUMENTO
%-----------------------------------------------%
\instituicao{Universidade do Estado de Santa Catarina -- UDESC}
\titulo{Estágio Curricular Supervisionado -- III}
\autor{Nome}
\local{Joinville - SC}
\data{Agosto/2022}
\tipotrabalho{Relatório}
\orientador{Prof. Dr. Orientador}
\coorientador{Prof. Me. Supervisor}
%-----------------------------------------------%
% Para alterar o parâmetros dos comandos orientador
% e coorientador.
%-----------------------------------------------%
% \renewcommand{\orientadorname}{Orientadora:}
\renewcommand{\coorientadorname}{Supervisor:}
%-----------------------------------------------%

\newcommand{\centro}{Centro de Ciências Tecnológicas -- CCT }
\newcommand{\departamento}{Departamento de Física -- DFIS}
\newcommand{\curso}{Licenciatura em Física }
\newcommand{\disciplina}{Estágio Curricular Supervisionado III -- ESC3003}
\newcommand{\firstkey}{Estágio Supervisionado}
\newcommand{\secondkey}{Ensino de Física}
\newcommand{\thirdkey}{Ensino Médio}


%-----------------------------------------------%

%	Todas as indicações de pacotes e configurações estão no arquivo de estilo
%  chamado texmodel-udesc.sty.
\usepackage{texmodel-udesc}

%-----------------------------------------------%
% Estilo de cabeçalho que só contém o número da 
% página e uma linha
%-----------------------------------------------%
\makepagestyle{cabecalholimpo}
\makeevenhead{cabecalholimpo}{\thepage}{}{} % páginas pares
\makeoddhead{cabecalholimpo}{}{}{\thepage} % páginas ímpares
%\makeheadrule{cabecalholimpo}{\textwidth}{\normalrulethickness} % linha
%-----------------------------------------------%

%-----------------------------------------------%
% Início do documento
%-----------------------------------------------%
\begin{document}
% cabecalho
\thispagestyle{empty}
\begin{center}
	\begin{minipage}[!]{\linewidth}
        \begin{minipage}[!]{.19\linewidth}
            \includegraphics[width=\linewidth]{img/logo.png}           
        \end{minipage}
        \begin{minipage}[!]{.8\linewidth}
            \center
            \ABNTEXchapterfont\normalsize\MakeUppercase{\imprimirinstituicao}
            \par
            \vspace*{10pt}                     
            \ABNTEXchapterfont\normalsize\MakeUppercase{\centro}
            \par
            \vspace*{10pt}           
            \ABNTEXchapterfont\normalsize\MakeUppercase{\disciplina}
        \end{minipage}        
    \end{minipage}
    \\ \vspace{0.5cm}
    \rule{\textwidth}{.5pt}   
\end{center}
    \textual
    \begin{center}
    \textbf{Plano de Aula: Intervenção Pedagógica nº 00X}
    \par\end{center}
    
    \noindent \textbf{Estagiário(a): }\imprimirautor 
    
    \noindent \textbf{U.E.: }EEB NOME DA ESCOLA
    
    \noindent \textbf{Série: }Xº Ano\hfill{}\textbf{Turma: }Xº--N
    
    \noindent \textbf{Aula:} 00X\hfill{}\textbf{Data:} XX/XX/2022\hfill{}\textbf{Duração:} $XX\min$
    \rule{\textwidth}{.5pt}
    \bigskip{}  
    

    \noindent \begin{center}
    \textbf{Título: Título da Aula}
    \par\end{center}

    \noindent \textbf{Resumo da aula: }

    \par\noindent \textbf{Habilidades BNCC: }EM13CNT101; EM13CNT301.
	
    \section{Objetivo de Aprendizagem}
    \begin{itemize}
        \item Perceber 
    \end{itemize}
    
    \medskip{}
    
    \noindent \textbf{Dimensão Conceitual:} \emph{Dimensão 01; Dimensão 02; Dimensão 03.}
    \newpage
    
    \section{Procedimento Didático} 
    \noindent \emph{1º Momento:} Título do primeiro momento.
	\par\noindent\rule{.3\textwidth}{.5pt}  
    \par\noindent \textbf{Tempo previsto:} XX minutos

    \noindent \textbf{Dinâmica:} Descrever a dinâmica do primeiro momento.

    \vspace{50pt}
    \noindent \emph{2º Momento:} Título do segundo momento.
	\par\noindent\rule{.3\textwidth}{.5pt}    
    \par\noindent \textbf{Tempo previsto: }XX minutos
	

    \noindent \textbf{Dinâmica:} Descrever a dinâmica do segundo momento.

	\vspace{50pt}
    \noindent \emph{3º Momento:} Título do terceiro momento.
	\par\noindent\rule{.3\textwidth}{.5pt}
    \par\noindent \textbf{Dinâmica:} Descrever a dinâmica do terceiro momento.
    
	\noindent Uma citação qualquer \cite{LAMBERTS:2011}, e uma equação em destaque 

    
	\begin{align}
		x^2&=ax-\nabla\vec{F}_\mu\nu
	\end{align}
	\bibliography{bibliografia.bib}
    \begin{anexosenv}		    
        \chapter{Sensibilidade Térmica}
        \section{Materiais}
        \begin{itemize}
            \item 3 recipientes (copos);
            \item 3 termômetros;
            \item água em temperatura ambiente;
            \item água aquecida ($40\Celsius\sim50\Celsius$);
            \item água fria ($15\Celsius\sim20\Celsius$).
        \end{itemize}
        \section{Procedimento Experimental}
        Coloque no primeiro recipiente água fria (com gelo); no segundo água à temperatura ambiente e no terceiro, água aquecida. Em seguida, coloque, ao mesmo tempo, uma mão na água gelada e a outra na água aquecida. Espere alguns instantes ($\sim5\min$) e coloque as duas mãos, ao mesmo tempo, no recipiente com água à temperatura ambiente.

        \subsection{Questões para debater}
        \begin{quest}
            Quando as suas mãos estavam em recipientes separados você podia distinguir qual continha água aquecida e qual continha água gelada?
        \end{quest}
        \begin{quest}
            Na situação descrita acima, é possível determinar a temperatura da água em algum dos recipientes?
        \end{quest}
        \begin{quest}
            Ao juntarmos as mãos no recipiente em que está a água à temperatura ambiente, continuou a ter as mesmas sensações?
        \end{quest}
        \begin{quest}
            Descreva o que sentiu.
        \end{quest}
    \end{anexosenv}
\end{document}
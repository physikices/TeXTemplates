%============| INÍCIO |=======================================>
%--------------| Q01 |---------------------------------------->
\addcontentsline{toc}{section}{Problema 01}
\begin{prob}
	Tanto a Lua quanto o Sol produzem as marés e, por estar mais próxima, a Lua desempenha
	um papel mais importante. Podemos concluir que a Lua atrai os oceanos com força gravitacional
	mais intensa do que o Sol? Justifique.

	\begin{sol}
		Sim! As marés são resultantes da atração gravitacional exercida pela Lua sobre a Terra e, também pelo Sol sobre a Terra, porém em menor escala devido as distâncias envolvidas, a força diferencial "cai" com o cubo da distância. É possível compará-las, utilizando
		\begin{align}
			dF=\frac{2GMm}{r^3}dr
			\label{eq:forcasDeMare}
		\end{align}
		para uma partícula de massa $m$ na superfície da Terra, temos que a razão entre estas duas forças é da ordem de
	\begin{align}
		\frac{dF_\odot}{dF_L}&=\frac{M_\odot}{M_L}\left(\frac{d_L}{d_\odot}\right)^3\\
		\frac{dF_\odot}{dF_L}&=\frac{1,989\times 10^{30}\kg}{7,35\times 10^{22}\kg}\left(\frac{3,84\times 10^5\km}{1,49\times 10^{8}\km}\right)\\
		dF_\odot&=0,46 dF_L 
	\end{align}
	\end{sol}
\end{prob}
%--------------| Q02 |---------------------------------------->
\addcontentsline{toc}{section}{Problema 02}
\begin{prob}
	Calcule a razão entre as forças de maré (máximas) no cometa Halley no afélio (35 UA) e
	periélio (0,59 UA). Quando se encontra no periélio, o cometa atingiu uma distância menor do que o limite de Roche? Considere que as densidades do cometa e do Sol sejam iguais.
	\begin{sol}
		Dada a \autoref{eq:forcasDeMare} considerando a distância do afélio $r_a$ e periélio $r_p$ tem-se
		\begin{align}
			\frac{dF_a}{dF_p}&=\frac{2GMmdr}{2GMmdr}\frac{r_p^3}{r_a^3}\\\nonumber
			\frac{dF_a}{dF_p}&=\left(\frac{r_p}{r_a}\right)^3\\\nonumber
			\frac{dF_a}{dF_p}&=\left(\frac{0,59\au}{35\au}\right)^3\\\nonumber
			\frac{dF_a}{dF_p}&=4,79\times 10^{-6}
		\end{align}
		A distância mínima $d$ dada pelo limite de Roche para satélites sólidos é 
		\begin{align}
			d&\leq 1,38\left(\frac{\rho_M}{\rho_m}\right)^{1/3}R
			\label{eq:limiteRoche}
		\end{align}
		Neste caso $\rho_m=\rho_M$ então
		\begin{align}
			d&\leq1,38R_\odot
		\end{align}
		Sendo o raio do sol $R_\odot=6,96\times 10^{5}\km$ e $0,59\au=8,8264\times 10^{7}\km$ então
		\begin{align}
			d&=1,38(6,96\times 10^{5}\km)\\\nonumber
			d&=9,6\times 10^5\km
		\end{align}
		protanto, sim! $d<0,59\au$.

	\end{sol}
\end{prob}
%--------------| Q03 |---------------------------------------->
\addcontentsline{toc}{section}{Problema 03}
\begin{prob}
	Calcule a razão entre a força gravitacional diferencial máxima da Lua sobre uma partícula
	de massa $m$ na superfície da Terra e a força auto - gravitacional da Terra sobre a mesma partícula de massa $m$ em sua superfície. De acordo com o resultado obtido, devemos considerar os efeitos da força gravitacional diferencial em experimentos de queda livre? Justifique.
	\begin{sol}
		A força de maré causada em uma partícula na Terra devida a interação gravitacional da Lua é
		\begin{align}
			dF_{L\to T}&=\frac{2GM_Lm}{d^3_{L\to T}}R_T
		\end{align}
		e a força auto-gravitacional da Terra, é dada por
		\begin{align}
			F_G&=\frac{GM_Tm}{R_T^2}
		\end{align}
		logo
		\begin{align}
			\frac{dF_{L\to T}}{F_G}&=\frac{2GM_Lm}{d^3_{L\to T}}\frac{R_T^2}{GM_Tm}R_T\\\nonumber
			\frac{dF_{L\to T}}{F_G}&=\frac{2M_L}{M_T}\left(\frac{R_T}{d_{L\to T}}\right)^3
		\end{align}
		se $M_T=5,97\times 10^{24}\kg$; $R_T=6,37\times 10^3\km$; $M_L=7,35\times 10^{22}\kg$ e $d_{L\to T}=3,84\times10^5\km$, substituindo temos
		\begin{align}
			\frac{dF_{L\to T}}{F_G}&=1,124\times 10^{-7}
		\end{align}
		Estes esfeitos apenas devem ser considerados nos casos em que necessita-se de uma precisão da ordem da sétima casa decimal, em geral, podem ser desprezados no exame dos movimentos de queda livre.
	\end{sol}
\end{prob}
%--------------| FIM |---------------------------------------->

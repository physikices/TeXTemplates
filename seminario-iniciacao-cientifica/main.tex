\documentclass[11pt]{article}
\usepackage{subcaption}
\usepackage{graphicx,float}
\usepackage{lipsum}
\usepackage[T1]{fontenc}
\usepackage[brazil]{babel}
\usepackage{authblk}
\renewcommand\Authand{, }
\renewcommand\Authands{, }
\usepackage{lmodern}		  	% Usa a fonte Latin Modern			
\usepackage{amsmath}
\usepackage{amssymb}
\usepackage{physics}
\usepackage{hyperref}
\usepackage[symbol]{footmisc}
\setfnsymbol{chicago}
\usepackage{physunits}

\usepackage{epsfig}
\usepackage{tikz}
\usetikzlibrary{decorations.markings}
\usetikzlibrary{shadows}
\usepackage[compat=1.1.0]{tikz-feynhand}
\setlength{\feynhandarrowsize}{3pt}
% Legendas
\usepackage[
  font=footnotesize,
  labelfont=bf
]{caption}

\usepackage[
  a4paper,
	top=25pt,
	bottom=25pt,
	left=3cm,
	right=2cm,
	headsep=10pt,
	headheight=42pt, % as per the warning by fancyhdr
	includehead,
	includefoot,
	heightrounded, % to avoid spurious underfull messages
	footskip=58pt,
]{geometry}

\usepackage[
  super,
	square,
]{natbib}
\setcitestyle{citesep={,}}
% \usepackage{showframe}

\hypersetup{
     	%pagebackref=true,
		colorlinks=true,       		% false: boxed links; true: colored links
		linkcolor=blue!40!black,          	% color of internal links
		citecolor=blue!40!black, 	% color of links to bibliography
		filecolor=magenta,      	% color of file links
		urlcolor=blue,
		bookmarksdepth=4
}

\usepackage{fancyhdr}
\pagestyle{fancy}
\fancyhead{} % clear all header fields
\fancyhead[LO]{\includegraphics[width=.2\linewidth]{img/cor_horizontal_ass_1_rgb.jpg}}
\fancyhead[RO]{\includegraphics[width=.35\linewidth]{img/sic.png}}
\fancyfoot{} % clear all footer fields
\fancyfoot{}
\fancyfoot[LO]{Apoio: \includegraphics[width=.2\linewidth]{img/cnpq.png}}
\fancyfoot[CO]{\includegraphics[width=.3\linewidth]{img/fapesc.png}}
\fancyfoot[RO]{pp.~\thepage}
% \fancyfoot[CO,RE]{p.~\thepage}

\title{\textbf{\uppercase{Produção de Mésons Vetoriais no Formalismo da Fatorização Colinear}}\textsuperscript{1}}


\author[2]{\small Rodrigo Ribamar Silva do Nascimento}
\author[3]{Bruno Duarte da Silva Moreira}

\affil[1]{\small Vinculado ao projeto: Estudo da Cromodinâmica Quântica no Regime de Altas Energias}
\affil[2]{Acadêmico do Curso de Licenciatura em Física -- CCT -- Bolsista PROBIC/UDESC}
\affil[3]{Orientador, Departamento de Física -- CCT -- bruno.moreira@udesc.br}




\usetikzlibrary{positioning}
\usetikzlibrary{shadows.blur, trees}

\begin{document}
\date{}
\maketitle
\thispagestyle{fancy}

O Modelo Padrão da Física de Partículas (SM)\footnote{Neste trabalho optamos por manter todas as siglas em suas respectivas versões do inglês.}, é atualmente a teoria amplamente aceita para a descrição adequada dos constituintes basilares da matéria ordinária e suas interações. Segundo o modelo, a matéria a nível microscópico pode ser entendida em termos de três classes de interações fundamentais: \textit{interações fortes, interações eletromagnética e interações fracas}; cada interação é descrita por uma Teoria Quântica de Campo localmente relativística \cite{Altarelli2020-ga}. Neste contexto, a Cromodinâmica Quântica (QCD) emerge como a Teoria Quântica responsável por descrever as interações fortes, em resumo, dentre todos os constituíntes da matéria, somente os quarks\footnote{Partículas elementares da família dos férmions, possuem spin $S=1/2$ e carga elétrica fracionária, combinam-se entre si para formar os hádrons como os bárions e mésons.} carregam carga de cor e podem interagir com a força forte por meio da troca de glúons\footnote{Partículas elementares da família dos bósons, possuem spin $S=1$, não possuem massa nem carga elétrica.}, diferentemente dos fótons que não carregam a sua carga de interação, os glúons são objetos bicolores portadores da carga cor e uma anticor, portanto, podem interagir entre si. Quarks e glúons jamais foram observados diretamente, somente em estados ligados (\textit{sem cor}), isto sugere que só podem existir em estados confinados formando os hádrons, esta propriedade é conhecida na QCD por \textit{confinamento}, por outro lado, a força de interação torna-se assintoticamente fraca com o aumento da energia e o decréscimo da distância entre estes constituintes, propriedade da QCD conhecida como \textit{liberdade assintótica}.

Ao longo dos últimos anos, o SM vem sendo testado e corroborado por diversos experimentos elevando-o ao \textit{approach} mais bem sucedido da Física até o momento \cite{GRIFFTHS:2008}, no entanto, o modelo encontra-se incompleto. As pesquisas dirigidas durante a atuação do colisor HERA obtiveram detalhes importantes acerca da estrutura dos prótons no regime de altas energias, possibilitando vincular satisfatoriamente o conteúdo de quarks de mar, todavia, ainda há elevada incerteza em relação ao conteúdo de glúons \cite{LUIS:2014}. Uma maneira de estudar a estrutura interna desses hádrons, vem sendo empregada em processos que envolvem a fotoprodução difrativa do méson vetorial $J/\psi$ \cite{RYSKIN:1996}, dado que sua massa fornece naturalmente uma escala dura de energia capaz de viabilizar o uso de métodos perturbativos da Cromodinâmica Quântica (pQCD). Segundo a QCD, no referido processo, um fóton $\gamma$ pode flutuar num par quark-antiquark pesado como, por exemplo, o charmonium $c\bar{c}$, dessa forma, interagem na troca de dois glúons $g$ produzidos no interior do hádron, cada qual portando diferentes frações de momento $l$ e $l^{\prime}$. O diagrama da \autoref{b} é usado no cálculo da seção de choque do processo $\gamma p\to pJ/\psi$ em função da energia do centro de massa $W$ do sistema, é esperado que para valores pequenos da variável $x$ de Bjorken, a seção de choque da fotoprodução difrativa do méson vetorial $J/\psi$, comporte-se de forma proporcional ao quadrado da distribuição de glúons $xg$ do próton alvo, tendo a sua forma predeterminada pela equação \eqref{eq:secao-de-choque}.

\begin{align}
	\sigma^{\gamma p\to J/\psi p} & = R_{g}^{2}(1+\beta^{2}) \frac{1}{b_{v}}\frac{\Gamma_{e^{+}e^{-}}M_{J/\psi}^{3} \pi^{3}}{48 \alpha_{em}}\frac{\alpha^{2}_{s}(\bar{Q}^{2})}{\bar{Q}^{8}}\left[xg(x,\bar{Q}^{2})\right]^{2}
	\label{eq:secao-de-choque}
\end{align}

Na eq \eqref{eq:secao-de-choque}, $R_{g}$ é o fator \textit{skeweness} referente a diferença das frações de momentum trocados pelos dois glúons durante a interação; o termo de correção da parte real do processo é determinado pela expressão $(1+\beta)^{2}$; o parâmetro de inclinação $b_{v}=4,5\;\textrm{GeV}^{2}$ é obtido por parametrização de ajustes experimentais; a amplitude de decaimento é indicada por $\Gamma_{e^{+}e^{-}}$; o valor da massa do méson $J/\psi$ é dado por $M_{J/\psi}=3,097\;\textrm{GeV}$; a constante de estrutura fina é conhecida e vale $\alpha_{em}=1/137$; já para a constante de acoplamento da QCD usamos $\alpha_{s}=0,2$; a escala de energia está relacionada à massa do méson da seguinte forma $\bar{Q}^{2}=M^{2}_{J/\psi}/4$; a razão entre a variável de Bjorken e a energia do centro de massa fica designada por $x=M_{J/\psi}^{2}/W^{2}$, por fim, a distribuição de glúons do alvo é calculada por $xg(x,\bar{Q}^{2})$.

Para utilizar os métodos da pQCD, é necessário o conhecimento preciso das distribuições partônicas\footnote{Distribuições de quarks e glúons} (PDFs), essas distribuições são obtidas pela solução das Equações DGLAP com os parâmetros vinculados experimentalmente por meio de análises globais, diferentes grupos de pesquisas produzem essas PDFs. Neste trabalho, utilizou-se especificamente os resultados da pesquisa conduzida\cite{LUIS:2014} em colaboração com os grupos CETQ(cteq6l)\footnote{The Coordinated Theoretical -- Experimental Project on QCD} e MRST, de posse destes dados implementou-se uma rotina em FORTRAN o que possibilitou a análise da seção de choque $\sigma^{\gamma p\to pJ/\psi}$ em função da energia do centro de massa estabelecida pela expressão \eqref{eq:secao-de-choque} e para a escala de energia indicada, obteve-se como resultados o gráfico da \autoref{a}. A partir do gráfico observa-se o  aumento da seção de choque com a energia do centro de massa, o que está de acordo com os resultados da pesquisa utilizada como base, dessa forma, pode-se estudar o comportamento da distribuição de glúons na fotoprodução do méson vetorial $J/\psi$ em regime de altas energias. No gráfico também fica claro a discrepância dos valores esperados para seção de choque do méson entre os grupos CTEQ e MRST, indicando incertezas relacionadas à estrutura interna do hádron. Em pesquisas futuras, esta análise pode ser extendida adicionando-se novas PDFs produzidas por outros grupos e a partir dos resultados, enriquecer nosso entendimento sobre a estrutura interna dos hádrons bem como as predições da QCD.

% O confronto e análise destes observáveis, podem melhor serem explorados em pesquisas futuras utilizando-se PDFs produzidas por outros grupos, a fim de ampliar nosso conhecimento sobre a estrutura dos hádrons e as predições da QCD.

	\begin{figure}[htb!]
		\centering
		\subfloat[\centering Diagrama de Feynmann para a fotoprodução do méson $J/\psi$ \label{b}]{
				\begin{tikzpicture}
					\input{img/jpsi_diagram-c.tex}
				\end{tikzpicture}
		}\qquad 
		\subfloat[\centering $\sigma^{\gamma p\to J/\psi p}$ em função da energia do centro de massa $W$, para escala de energia $\mu^{2}=2,4\;\textrm{GeV}^2$ \label{a}]{
			\includegraphics[width=.4\linewidth]{img/jpsi.eps}
		}
		\caption{\textit{Fotoprodução do méson $J/\psi$ via troca de dois glúons}}
		\label{fig:a}
	\end{figure}

\vspace{10pt}
\begin{center}
	\parbox{.9\textwidth}{\noindent\textbf{\small Palavras-chave:} \small Cromodinâmica Quântica. Fotoprodução do $J/\psi$. Fatorização Colinear.}
\end{center}
\vspace{20pt}

% --------------------------------------------- %
\bibliographystyle{acm}
\bibliography{referencias}
% --------------------------------------------- %
\end{document}



%-----------------------------------------------%
% Modelo de Plano de Aula com três momentos pedagógicos
%
% Autor: Rodrigo Nascimento (2022-08-12)
%-----------------------------------------------%

\documentclass[
% -- opções da classe memoir --
12pt,				% tamanho da fonte
openright,			% capítulos começam em pág ímpar (insere página vazia caso preciso)
oneside,			% twoside para impressão em verso e anverso. Oposto a oneside
a4paper,			% tamanho do papel. 
% -- opções da classe abntex2 --
chapter=TITLE,		% títulos de capítulos convertidos em letras maiúsculas
%section=TITLE,		% títulos de seções convertidos em letras maiúsculas
%subsection=TITLE,	% títulos de subseções convertidos em letras maiúsculas
%subsubsection=TITLE,% títulos de subsubseções convertidos em letras maiúsculas
% -- opções do pacote babel --
english,			% idioma adicional para hifenização
%	french,				% idioma adicional para hifenização
%	spanish,			% idioma adicional para hifenização
brazil				% o último idioma é o principal do documento
]{abntex2}
\selectlanguage{brazil}
%-----------------------------------------------%
% Informações do DOCUMENTO
%-----------------------------------------------%
\instituicao{Universidade do Estado de Santa Catarina -- UDESC}
\titulo{Estágio Curricular Supervisionado -- III}
\autor{Nome}
\local{Joinville - SC}
\data{Agosto/2022}
\tipotrabalho{Relatório}
\orientador{Prof. Dr. Orientador}
\coorientador{Prof. Me. Supervisor}
%-----------------------------------------------%
% Para alterar o parâmetros dos comandos orientador
% e coorientador.
%-----------------------------------------------%
% \renewcommand{\orientadorname}{Orientadora:}
\renewcommand{\coorientadorname}{Supervisor:}
%-----------------------------------------------%

\newcommand{\centro}{Centro de Ciências Tecnológicas -- CCT }
\newcommand{\departamento}{Departamento de Física -- DFIS}
\newcommand{\curso}{Licenciatura em Física }
\newcommand{\disciplina}{Estágio Curricular Supervisionado III -- ESC3003}
\newcommand{\firstkey}{Estágio Supervisionado}
\newcommand{\secondkey}{Ensino de Física}
\newcommand{\thirdkey}{Ensino Médio}


%-----------------------------------------------%

%	Todas as indicações de pacotes e configurações estão no arquivo de estilo
%  chamado texmodel-udesc.sty.
\usepackage{texmodel-udesc}

%-----------------------------------------------%
% Estilo de cabeçalho que só contém o número da 
% página e uma linha
%-----------------------------------------------%
\makepagestyle{cabecalholimpo}
\makeevenhead{cabecalholimpo}{\thepage}{}{} % páginas pares
\makeoddhead{cabecalholimpo}{}{}{\thepage} % páginas ímpares
%\makeheadrule{cabecalholimpo}{\textwidth}{\normalrulethickness} % linha
%-----------------------------------------------%

%-----------------------------------------------%
% HEADER
%-----------------------------------------------%
\begin{document}

\thispagestyle{empty}
\begin{center}
	\begin{minipage}[!]{\linewidth}
		\begin{minipage}[!]{.19\linewidth}
			\includegraphics[width=\linewidth]{img/logo.png}           
		\end{minipage}
		\begin{minipage}[!]{.8\linewidth}
			\center
			\ABNTEXchapterfont\normalsize\MakeUppercase{\imprimirinstituicao}
			\par
			\vspace*{10pt}                     
			\ABNTEXchapterfont\normalsize\MakeUppercase{\centro}
			\par
			\vspace*{10pt}           
			\ABNTEXchapterfont\normalsize\MakeUppercase{\disciplina}
		\end{minipage}        
	\end{minipage}
	\\ \vspace{0.5cm}
	\rule{\textwidth}{.5pt}   
\end{center}
%-----------------------------------------------%
% Ficha de Identificação
%-----------------------------------------------%
\textual
\begin{center}
	\textbf{Plano de Aula: Intervenção Pedagógica nº 00X}
\end{center}
\par\noindent\textbf{Estagiário(a):} \imprimirautor
\par\noindent\textbf{U.E.:} EEB Giovani Pasqualini Faraco
\par\noindent\textbf{Série:} Xº Ano\hfill{}\textbf{Turma:} Xº--N
\par\noindent\textbf{Aula:} 001\hfill{}\textbf{Data:} \mydate\hfill{}\textbf{Duração:} $45\min$
\rule{\textwidth}{.5pt}
%-----------------------------------------------%
% Início do Plano de Aula
%-----------------------------------------------%
\bigskip{}  
\noindent
\begin{center}
	\textbf{Título: Título da Aula}
\end{center}
\par\noindent\textbf{Resumo da aula: }
\par\noindent\textbf{Habilidades BNCC: }EM13CNT101; EM13CNT301.

\section{Objetivo de Aprendizagem}
\begin{itemize}
	\item Perceber
\end{itemize}

\medskip{}

\noindent\textbf{Dimensão Conceitual:} \emph{Dimensão 01; Dimensão 02; Dimensão 03.}
\newpage

\section{Procedimento Didático} 
\noindent\emph{1º Momento:} Título do primeiro momento.
\par\noindent\rule{.3\textwidth}{.5pt}  
\par\noindent\textbf{Tempo previsto:} XX minutos

\noindent\textbf{Dinâmica:} Descrever a dinâmica do primeiro momento.

\vspace{50pt}
\noindent\emph{2º Momento:} Título do segundo momento.
\par\noindent\rule{.3\textwidth}{.5pt}    
\par\noindent\textbf{Tempo previsto: }XX minutos


\noindent\textbf{Dinâmica:} Descrever a dinâmica do segundo momento.

\vspace{50pt}
\noindent\emph{3º Momento:} Título do terceiro momento.
\par\noindent\rule{.3\textwidth}{.5pt}
\par\noindent\textbf{Dinâmica:} Descrever a dinâmica do terceiro momento.
%-----------------------------------------------%
% Referências
%-----------------------------------------------%
% \bibliography{bibliografia.bib}
%-----------------------------------------------%
% Anexos
%-----------------------------------------------%
% \begin{anexosenv}		    
% \end{anexosenv}
%-----------------------------------------------%
% Fim do Plano de Aula
%-----------------------------------------------%
\end{document}

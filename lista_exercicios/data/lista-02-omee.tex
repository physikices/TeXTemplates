% ============| INÍCIO |===================
%--------------| Q01 |--------------------
\addcontentsline{toc}{section}{Problema 01}
\begin{prob}
  A energia de um sistema de $N$ íons magnéticos localizados, a temperatura $T$, na presença de um campo magnético $H$, pode ser escrita na forma
  \begin{align}
    \mathcal{H}&=D\sum_{i=1}^NS_i^2-\mu_0H\sum_{i=1}^NS_i
  \end{align}
  onde os parâmetros $D$, $\mu_0$ e $H$ são positivos e $S_i=+1,0,-1$, para qualquer sítio $i$. Obtenha:
  \begin{enumerate}[label=\alph *)]
    \item A função de partição do sistema.
    \item A energia interna.
    \item A entropia.
  \end{enumerate}
  \begin{sol}
    Dado que o sistema de interesse, está em contato térmico com o reservatório a temperatura $T$ (assumidamente fixa), podemos utilizar a função de partição definida para o Ensemble Canônico
    \begin{align}
      Z&=\sum_j\E^{-\beta E_j}
    \end{align}
    em que $E_j$ é a energia do sistema, no $j-$ésimo estado microscópico particular. Considerando íons indistinguíveis, ou não interagentes, se conhecermos a função de partição de partícula única $Z_1$, basta fazer $Z_1^N$ que conheceremos a função de partição total $Z$. Assumindo estas considerações, prosseguimos:
    \begin{enumerate}[label=\alph *)]
      \item Determinando a função de partição $Z$, a partir da função de partição de partícula única $Z_1$
      \begin{align}
        \begin{split}
          Z_1&=\sum_{S_1=-1,0+1}\E^{-\beta\mathcal{H}_1}\\
          Z_1&=\sum_{S_1=-1,0+1}\E^{-\beta\left(DS_1^{2}-\mu_0HS_1\right)}\\
          Z_1&=\E^{-\beta\left[D(-1)^2-\mu_0H(-1)\right]}+\E^{-\beta\left[D(0)^2-\mu_0H(0)\right]}+\\
          &\quad+\E^{-\beta\left[D(1)^2-\mu_0H(1)\right]}\\
          Z_1&=1+\E^{-\beta\left(D+\mu_0H\right)}+\E^{-\beta\left(D-\mu_0H\right)}\\
          Z_1&=1+\E^{-\beta D}\E^{-\beta\mu_0H}+\E^{-\beta D}\E^{\beta\mu_0H}\\
          Z_1&=1+\E^{-\beta D}\left(\E^{-\beta\mu_0H}+\E^{\beta\mu_0H}\right)\\
          Z_1&=1+2\E^{-\beta D}\cosh\beta\mu_0H       
        \end{split}
      \end{align}
      logo,
      \begin{align}
        \boxed{
            Z=\left(1+2\E^{-\beta D}\cosh\beta\mu_0H \right)^N
          }
      \end{align}
      \item A partir da função de partição $Z$, podemos obter a energia interna $U$ dada simplesmente por
      \begin{align}
        U&=-\parder{}{\beta}\ln Z=-N\parder{}{\beta}\ln Z_1
      \end{align}
      portanto,
      \begin{align}
        \begin{split}
          U=-N\parder{}{\beta}\ln\Bigl(&1+2\E^{-\beta D}\cosh\beta\mu_0H\Bigr)\\
          U=-\frac{N}{1+2\E^{-\beta D}\cosh\beta\mu_0H}\Biggl[&\parder{}{\beta}\left(1+2\E^{-\beta D}\cosh\beta\mu_0H\right)\Biggr]\\
          U=-\frac{N}{1+2\E^{-\beta D}\cosh\beta\mu_0H}\Bigl[&-2D\E^{-\beta D}\cosh\beta\mu_0H+\\
          &+2\mu_0H\E^{-\beta D}\senh\beta\mu_0H\Bigr]          
        \end{split}
      \end{align}
      \begin{align}
        \boxed{
          U=\frac{2N\E^{-\beta D}}{1+2\E^{-\beta D}\cosh\beta\mu_0H}\Bigl[D\cosh\beta\mu_0H-\mu_0H\senh\beta\mu_0H\Bigr]
        }
      \end{align}
      \item A entropia $S$ do sistema, é obtida pela energia livre de Helmholtz $F$, bastando-se fazer 
      \begin{align}
        S=-\parder{F}{T}
      \end{align}
      em que $F$ em termos da função de partição $Z$ é simplesmente
      \begin{align}
        \begin{split}
          F&=-\frac{1}{\beta}\ln Z\\
          F&=-N\frac{1}{\beta}\ln\left(1+2\E^{-\beta D}\cosh\beta\mu_0H\right)\\          
        \end{split}
      \end{align}
      Matematicamente, se $F=F(\beta (T))$, então
      \begin{align}
        \label{eq:entropia-prob-1}
        -S&=\parder{F}{T}=\parder{F}{\beta}\frac{\dif\beta}{\dif T}
      \end{align}
      Avaliando $\partial_\beta F$:
      \begin{align}
        \label{eq:hemlholtz-prob-1}
        \begin{split}
          \parder{F}{\beta}&=-N\parder{}{\beta}\left[\frac{1}{\beta}\ln\left(1+2\E^{-\beta D}\cosh\beta\mu_0H\right)\right]\\
          \parder{F}{\beta}&=-N\left(-\frac{1}{\beta^2}\right)\ln\left(1+2\E^{-\beta D}\cosh\beta\mu_0H\right)+\\
          &\qquad-N\frac{1}{\beta}\left[\frac{-2D\E^{-\beta D}\cosh\beta\mu_0H+2\E^{-\beta D}\mu_0H\senh\beta\mu_0H}{1+2\E^{-\beta D}\cosh\beta\mu_0H}\right]\\
          \parder{F}{\beta}&=\frac{N}{\beta^2}\ln\left(1+2\E^{-\beta D}\cosh\beta\mu_0H\right)+\\
          &\qquad+\frac{2N\E^{-\beta D}}{\beta}\left[\frac{D\cosh\beta\mu_0H-\mu_0H\senh\beta\mu_0H}{1-2\E^{-\beta D}\cosh\beta\mu_0H}\right]
        \end{split}
      \end{align}
      Avaliando $\dif\beta/\dif T$, sendo $\beta=1/k_BT$ é direto que
      \begin{align}
        \label{eq:derivada-beta-prob-1}
        \frac{\dif\beta}{\dif T}&=-\frac{1}{k_BT^2}
      \end{align}
      escrevendo a entropia do sistema em função da temperatura a partir das equações \eqref{eq:entropia-prob-1}, \eqref{eq:hemlholtz-prob-1} e \eqref{eq:derivada-beta-prob-1} tem-se      
      \begin{align}
        \begin{split}
          -S&=\Biggl\{Nk_B^2T^2\ln\left[1+2\E^{-D/k_BT}\cosh\left(\mu_0H/k_BT\right)\right]+2Nk_BT\E^{-D/k_BT}\times\\
           &\qquad\times\left[\frac{D\cosh\left(\mu_0H/k_BT\right)-\mu_0H\senh\left(\mu_0H/k_BT\right)}{1-2\E^{-D/k_BT}\cosh\left(\mu_0H/k_BT\right)}\right]\Biggr\}\left(-\frac{1}{k_BT^2}\right)\\
        \end{split}
      \end{align}
      \begin{align}
        \addtolength{\fboxsep}{5pt}
         \boxed{
         \begin{gathered}
          S=Nk_B\ln\left[1+2\E^{-D/k_BT}\cosh\left(\mu_0H/k_BT\right)\right]+\frac{2N\E^{-D/k_BT}}{T}\times\\
          \times\left[\frac{D\cosh\left(\mu_0H/k_BT\right)-\mu_0H\senh\left(\mu_0H/k_BT\right)}{1-2\E^{-D/k_BT}\cosh\left(\mu_0H/k_BT\right)}\right]        
         \end{gathered}
         }
      \end{align}
    \end{enumerate}  
  \end{sol}
\end{prob}
% --------------| Q02 |--------------------
\addcontentsline{toc}{section}{Problema 02}
\begin{prob}
  Em aula foi obtida a função de partição para um sistema 1D de sistemas magnéticos de $N$ spins localizados, a temperatura $T$, associados com a energia
  \begin{align}
    \mathcal{H}&=-J\sum_{i=1,3,5,...,N-1}\sigma_i\sigma_{i+1}-\mu_0H\sum_{i=1}^N\sigma_i
  \end{align}
  onde os parâmetros $J$, $\mu_0$ e $H$ são positivos, e para todos os estados $i$. Além disso, $N$ é um número par e a primeira soma corre para números ímpares.
  \begin{enumerate}[label=\alph *)]
    \item Calcule a energia interna por partícula $u(T,H)$. Esboce um gráfico de $u(T,H=0)$.
    \item Calcule a entropia por partícula $s(T,H)$. Esboce um gráfico de $s(T,H=0)$.
    \item Obtenha expressões para a magnetização por partícula $m(T,H)$ e para a suscetibilidade magnética
  \end{enumerate}
  \begin{sol}
    A função de partição encontrada em aula é a seguinte
    \begin{align}
     Z&=\left[2\E^{\beta J}\cosh\left(2\beta\mu_0H\right)+2\E^{-\beta J}\right]^{N/2}
    \end{align}
    \begin{enumerate}[label=\alph *)]
      \item Determinando a energia interna por partícula $u$, a partir da função de partição $Z$
      \begin{align}
        u&=-\frac{1}{N}\parder{}{\beta}\ln Z
      \end{align}
      tem-se que
      \begin{align}
        \begin{split}
          u&=-\frac{1}{N}\frac{N}{2}\parder{}{\beta}\ln\left[2\E^{\beta J}\cosh\left(2\beta\mu_0H\right)+2\E^{-\beta J}\right]\\
          u&=-\frac{1}{2}\Biggl[\frac{2J\E^{\beta J}\cosh\left(2\beta\mu_0H\right)+4\mu_0H\E^{\beta J}\senh\left(2\beta\mu_0H\right)-2J\E^{-\beta J}}{2\E^{\beta J}\cosh\left(2\beta\mu_0H\right)+2\E^{-\beta J}}\Biggr]
        \end{split}
      \end{align}
      \begin{align}
        \addtolength{\fboxsep}{5pt}
         \boxed{
         \begin{gathered}
          u=\frac{J}{2}\left[\frac{1-\E^{2\beta J}\cosh\left(2\beta\mu_0H\right)-2J^{-1}\mu_0H\E^{2\beta J}\senh\left(2\beta\mu_0H\right)}{\E^{2\beta J}\cosh\left(2\beta\mu_0H\right)+1}\right]       
         \end{gathered}
        }
      \end{align}
      para $H=0$ $u(T)$ fica
      \begin{align}
        \begin{split}
          u&=\frac{J}{2}\left[\frac{1-\E^{2\beta J}\CancelTo[\color{deepgreen}]{1}{\cosh\left(2\beta\mu_0H\right)}-2J^{-1}\mu_0\CancelTo[\color{deepred}]{0}{H}\E^{2\beta J}\CancelTo[\color{deepred}]{0}{\senh\left(2\beta\mu_0H\right)}}{\E^{2\beta J}\CancelTo[\color{deepgreen}]{1}{\cosh\left(2\beta\mu_0H\right)}+1}\right]          
        \end{split}
      \end{align}
      \begin{align}
        \label{eq:energia-prob2a}
        \addtolength{\fboxsep}{5pt}
         \boxed{
         \begin{gathered}
            u(T)=-\frac{J}{2}\left(\frac{\E^{2J/k_BT}-1}{\E^{2J/k_BT}+1}\right)     
         \end{gathered}
        }      
      \end{align}
      A \autoref{fig:plot-prob2a} a seguir, representa o gráfico da energia por partícula em função da temperatura, quando $H=0$
      \begin{figure}[!ht]        
        \begin{center}
          \begin{tikzpicture} 
            \begin{axis}[
                axis lines = left,                
                xmin = 0, xmax = 10,
                ymin = -0.6, ymax = 0,
                xlabel = \(T\),
                ylabel = {\(u(T)\)},
                ylabel style={rotate=-90},
                ytick = {-0.5, 0},
                yticklabels = {$-\frac{J}{2}$,$0$},
                xtick = {0},
                %xticklabels = {$0$,$T_1$},
                legend pos = south east,
              ]
                \addplot[
                    domain = 0:100,
                    samples = 1000,
                    smooth,
                    thick,
                    javapurple,
                ] {(-1/2)*((exp(2/x)-1)/(exp(2/x)+1))};                
                \addlegendentry{\(u(T)=-\frac{J}{2}\left(\frac{\E^{2J/k_BT}-1}{\E^{2J/k_BT}+1}\right)\)}
                \addplot[
                    domain = 0:100,
                    samples = 1000,
                    dashed,
                    thick,
                    gray0.5,
                ] {0};
            \end{axis}         
          \end{tikzpicture}  
        \end{center}        
        \caption{Gráfico da energia interna por partícula $u(T,H=0).$}
        \label{fig:plot-prob2a}       
      \end{figure}
      \item A entropia por partícula $s$, é encontrada a partir da energia livre de Helmholtz $f$ como segue
      \begin{dmath*}
        s=-\parder{f}{T} \condition{com $f=-\displaystyle\frac{1}{\beta}\lim_{N\to\infty}\frac{1}{N}\ln Z$}
      \end{dmath*}
      ou seja
      \begin{align}
        \begin{split}
          f&=-\frac{1}{\beta}\lim_{N\to\infty}\frac{1}{N}\frac{N}{2}\ln\left[2\E^{\beta J}\cosh\left(2\beta\mu_0H\right)+2\E^{-\beta J}\right]\\
          f&=-\frac{1}{2\beta}\ln\left[2\E^{\beta J}\cosh\left(2\beta\mu_0H\right)+2\E^{-\beta J}\right]
        \end{split}
      \end{align}
      mas $f(\beta(T))$, então
      \begin{align}
        -s&=\parder{f}{T}=\parder{f}{\beta}\frac{\dif\beta}{\dif T}
      \end{align}
      calculando $\partial_\beta f$ primeiramente
      \begin{align}
        \begin{split}
          \parder{f}{\beta}&=\frac{1}{2\beta^2}\ln\left[2\E^{\beta J}\cosh\left(2\beta\mu_0H\right)+2\E^{-\beta J}\right]+\\&\qquad-\frac{1}{2\beta}\left[\frac{2J\E^{\beta J}\cosh\left(2\beta\mu_0H\right)+2\E^{\beta J}\left(2\mu_0H\right)\senh(2\beta\mu_0H)-2J\E^{-\beta J}}{2\E^{\beta J}\cosh\left(2\beta\mu_0H\right)+2\E^{-\beta J}}\right]
        \end{split}
      \end{align}
      a drivada de $\beta$ em relação à $T$ é simples e vale $d\beta/dT=-1/k_BT^2$. De modo que
      \begin{align}
        \begin{split}
          \parder{f}{T}&=-\frac{\beta}{2T\beta^2}\ln\left[2\E^{\beta J}\cosh\left(2\beta\mu_0H\right)+2\E^{-\beta J}\right]+\\&\qquad+\frac{2J\beta\E^{-\beta J}}{2T\beta 2\E^{-\beta J}}\left[\frac{\E^{2\beta J}\cosh\left(\beta\mu_0H\right)+2J^{-1}\mu_0H\E^{2\beta J}\senh\left(2\beta\mu_0H\right)-1}{\E^{2\beta J}\cosh\left(2\beta\mu_0H\right)+1}\right]\\
          \parder{f}{T}&=-\frac{k_BT}{2T}\ln\left[2\E^{J/k_BT}\cosh\left(2\mu_0H/k_BT\right)+2\E^{-J/k_BT}\right]+\\&\qquad+\frac{J}{2T}\left[\frac{\E^{2J/k_BT}\cosh\left(\mu_0H/k_BT\right)+2J^{-1}\mu_0H\E^{2J/k_BT}\senh\left(2\mu_0H/k_BT\right)-1}{\E^{2J/k_BT}\cosh\left(2\mu_0H/k_BT\right)+1}\right]\\
          \parder{f}{T}&=-\frac{k_B}{2}\ln\left[2\E^{J/k_BT}\cosh\left(2\mu_0H/k_BT\right)+2\E^{-J/k_BT}\right]+\\&\qquad+\frac{J}{2T}\left[\frac{\E^{2J/k_BT}\cosh\left(\mu_0H/k_BT\right)+2J^{-1}\mu_0H\E^{2J/k_BT}\senh\left(2\mu_0H/k_BT\right)-1}{\E^{2J/k_BT}\cosh\left(2\mu_0H/k_BT\right)+1}\right]
        \end{split}
      \end{align}
      logo
      \begin{align}
        \label{eq:entropia-prob2b}
        \addtolength{\fboxsep}{5pt}
         \boxed{
         \begin{gathered}
            s=\frac{k_B}{2}\ln\left[2\E^{J/k_BT}\cosh\left(2\mu_0H/k_BT\right)+2\E^{-J/k_BT}\right]+\\-\frac{J}{2T}\left[\frac{\E^{2J/k_BT}\cosh\left(\mu_0H/k_BT\right)+2J^{-1}\mu_0H\E^{2J/k_BT}\senh\left(2\mu_0H/k_BT\right)-1}{\E^{2J/k_BT}\cosh\left(2\mu_0H/k_BT\right)+1}\right]    
         \end{gathered}
        }      
      \end{align}
      se $H=0$ obtemos
      \begin{align}
        \begin{split}
          s&=\frac{k_B}{2}\ln\left[2\E^{J/k_BT}\CancelTo[\color{deepgreen}]{1}{\cosh\left(2\mu_0H/k_BT\right)}+2\E^{-J/k_BT}\right]+\\&\qquad-\frac{J}{2T}\left[\frac{\E^{2J/k_BT}\CancelTo[\color{deepgreen}]{1}{\cosh\left(\mu_0H/k_BT\right)}+2J^{-1}\mu_0\CancelTo[\color{deepred}]{0}{H}\E^{2J/k_BT}\CancelTo[\color{deepred}]{0}{\senh\left(2\mu_0H/k_BT\right)}-1}{\E^{2J/k_BT}\CancelTo[\color{deepgreen}]{1}{\cosh\left(2\mu_0H/k_BT\right)}+1}\right]\\
          s&=\frac{k_B}{2}\ln\left[2\E^{-J/k_BT}\left(\E^{2J/k_BT}+1\right)\right]-\frac{J}{2T}\left(\frac{\E^{2J/k_BT}-1}{\E^{2J/k_BT}+1}\right)\\
          s&=\frac{k_B}{2}\left[\ln 2+\ln\E^{-J/k_BT}+\ln\left(\E^{2J/k_BT}+1\right)\right]-\frac{J}{2T}\left(\frac{\E^{2J/k_BT}-1}{\E^{2J/k_BT}+1}\right)\\
          s&=\frac{k_B}{2}\ln 2 + \frac{k_B}{2}\ln\left(\E^{2J/k_BT}+1\right)-\frac{J}{2T}\left[1+\frac{\E^{2J/k_BT}-1}{\E^{2J/k_BT}+1}\right]
        \end{split}
      \end{align}
      \begin{align}
        \label{eq:entropia-prob2a}
        \addtolength{\fboxsep}{5pt}
         \boxed{
         \begin{gathered}
            s(T)=\frac{k_B}{2}\ln 2 + \frac{k_B}{2}\ln\left(\E^{2J/k_BT}+1\right)-\frac{J}{T}\left[\frac{\E^{2J/k_BT}}{\E^{2J/k_BT}+1}\right]    
         \end{gathered}
        }    
      \end{align}
      O gráfico da entropia por partícula como função da temperatura pode ser visto na \autoref{fig:plot-prob2b} na sequência
      \begin{figure}[!ht]
        \begin{center}
          \begin{tikzpicture} 
            \begin{axis}[
                axis lines = left,                
                xmin = 0, xmax = 5,
                ymin = 0.3, ymax = 0.75,
                xlabel = \(T\),
                ylabel = {\(s(T)\)},
                ylabel style={rotate=-90},
                ytick = {0.3466,0.682},
                yticklabels = {$k_B\frac{\ln 2}{2}$,$k_B\ln 2$},
                xtick = {0},
                %xticklabels = {$0$,$T_1$},
                legend pos = south east,
              ]
                \addplot[
                    domain = -1:100,
                    samples = 1000,
                    smooth,
                    thick,
                    javapurple,
                ] {((1/2)*ln((2*exp(1/x))+(2*exp(-1/x))))-(1/(2*x))*((2*exp(2/x)-1)/(2*exp(2/x)+1))};                
                \addlegendentry{\(s(T)\)}
                \addplot[
                    domain = 0:100,
                    samples = 1000,
                    dashed,
                    thick,
                    gray0.5,
                ] {0.682};
            \end{axis}         
          \end{tikzpicture}
        \end{center}
        \caption{Gráfico da entropia por partícula em função da temperatura $u(T,H=0).$}
        \label{fig:plot-prob2b}
      \end{figure}
      \item A expressão para a magnetização por partícula $m$, é obtida através da função energia livre magnética por partícula $g(T,H)$, assim 
      \begin{dmath*}
        m=-\left(\parder{g}{H}\right)_T \condition{com $g(T,H)=-\displaystyle\frac{1}{\beta}\lim_{N\to\infty}\frac{1}{N}\ln Z$}
      \end{dmath*}
      Obtendo a $g(T,H)$
      \begin{align}
        \begin{split}
          g(T,H)&=-\frac{1}{\beta}\lim_{N\to\infty}\frac{1}{N}\ln Z\\
          g(T,H)&=-\frac{1}{\beta}\lim_{N\to\infty}\frac{1}{N}\ln \left[2\E^{\beta J}\cosh\left(2\beta\mu_0H\right)+2\E^{-\beta J}\right]^{N/2}\\
          g(T,H)&=-\frac{1}{\beta}\lim_{N\to\infty}\frac{1}{N}\frac{N}{2}\ln \left[2\E^{\beta J}\cosh\left(2\beta\mu_0H\right)+2\E^{-\beta J}\right]\\
          g(T,H)&=-\frac{1}{2\beta}\ln \left[2\E^{\beta J}\cosh\left(2\beta\mu_0H\right)+2\E^{-\beta J}\right]
        \end{split}
      \end{align}
      Obtendo a magnetização por partícula
      \begin{align}
        \begin{split}
          -m&=\parder{g(T,H)}{H}\\
          -m&=-\frac{1}{2\beta}\parder{}{H}\left(\ln \left[2\E^{\beta J}\cosh\left(2\beta\mu_0H\right)+2\E^{-\beta J}\right]\right)\\
          m&=\frac{1}{2\beta}\frac{1}{2\E^{-\beta J}}\left[\frac{2\E^{\beta J}\left(2\beta\mu_0\right)\senh\left(2\beta\mu_0H\right)}{\E^{2\beta J}\cosh\left(2\beta\mu_0H\right)+1}\right]\\
        \end{split}
      \end{align}
      \begin{align}
        \addtolength{\fboxsep}{5pt}
         \boxed{
         \begin{gathered}
          m(T,H)=\mu_0\left[\frac{\E^{2J/k_BT}\senh\left(2\mu_0H/k_BT\right)}{\E^{2J/k_BT}\cosh\left(2\mu_0H/k_BT\right)+1}\right]               
         \end{gathered}
        }    
      \end{align}
      e para a suscetibilidade magnética $\chi(T,H)$ tem-se que
      \begin{align}
        \chi(T,H)&=\left(\parder{m}{H}\right)_T
      \end{align}
      logo     
     \begin{align}
        \begin{split}
	        \chi&=\mu_0\parder{}{H}\left[\frac{\E^{2\beta J}\senh\left(2\beta\mu_0H\right)}{\E^{2\beta J}\cosh\left(2\beta\mu_0H\right)+1}\right]\\
	        \chi&=\frac{\mu_0}{\left[\E^{2\beta J}\cosh\left(\beta\mu_0H\right)+1\right]^2}\Bigg\{\\
          &\qquad\E^{2\beta J}\left(2\beta\mu_0\right)\cosh\left(2\beta\mu_0H\right)\biggl[\E^{2\beta J}\cosh\left(2\beta\mu_0H\right)+1\biggr]+\\
          &\qquad-\E^{2\beta J}\senh\left(\beta\mu_0H\right)\biggl[\left(2\beta\mu_0H\right)\senh\left(\beta\mu_0H\right)\biggr]\Bigg\}\\
          \chi&=\frac{2\E^{2\beta J}\mu_0^2\beta}{\left[\E^{2\beta J}\cosh\left(\beta\mu_0H\right)+1\right]^2}\biggl[\E^{2\beta J}\cosh^2\left(\beta\mu_0H\right)+\\
          &\qquad-\E^{2\beta J}\senh(\beta\mu_0H)+\cosh\left(\beta\mu_0H\right)\biggr]\\
          \chi&=2\E^{2\beta J}\mu_0^2\beta\Biggl[\frac{\E^{2\beta J}+\cosh\left(\beta\mu_0H\right)}{\left[\E^{2\beta J}\cosh\left(\beta\mu_0H\right)+1\right]^2}\Biggr]
	      \end{split}
      \end{align}
      \begin{align}
        \addtolength{\fboxsep}{5pt}
         \boxed{
         \begin{gathered}
          \chi(T,H)=\frac{2\E^{4J/k_BT}\mu_0^2}{k_BT}\Biggl[\frac{\E^{-2J/k_BT}\cosh\left(\mu_0H/k_BT\right)+1}{\left[\E^{2J/k_BT}\cosh\left(\mu_0H/k_BT\right)+1\right]^2}\Biggr]               
         \end{gathered}
        }
      \end{align}
    \end{enumerate}
  \end{sol}
\end{prob}
% --------------| Q03 |--------------------
\addcontentsline{toc}{section}{Problema 03}
\begin{prob}
  Considere um sistema de $N$ partículas clássicas não interagentes em contato com um reservatório térmico a temperatura $T$. Cada partícula pode ter energia $0$, $\varepsilon>0$ e $3\varepsilon$.
  \begin{enumerate}[label=\alph *)]
    \item Obtenha uma expressão para a função canônica de partição.
    \item Calcule a energia interna por partícula $u(T)$. Calcule os limites de u(T) em $T\to0$ e $T\to\infty$.
    \item Calcule a entropia por partícula $s(T)$.
    \item Calcule o calor específico e esboce um gráfico de $c$ em função da temperatura.
  \end{enumerate}
  \begin{sol}
    Dado que a função de partição será calculada por
    \begin{align}
      Z&=\sum_{j=1}^N\E^{-\beta E_j}
    \end{align}
    \begin{enumerate}[label=\alph *)]
      \item É preciso encontrar uma expressão para a energia total do sistema $E_j$. Uma vez que o sistema de $N$ partículas não interagentes pode assumir três níveis de energias, definimos a variável $t_j$, associada à $j-\textrm{ésima}$ partícula que rotula os estados possíveis, de maneira que      
      \begin{align}
        t_j=
        \begin{cases}
          0,& \textrm{se }\varepsilon_1=0\\
          1,& \textrm{se }\varepsilon_2=\varepsilon>0\\
          3,& \textrm{se }\varepsilon_3=3\varepsilon
        \end{cases}
      \end{align}
      Um estado microscópico do sistema fica então caracterizado pelo conjunto de valores ${t_j}$, com energias tal que
      \begin{align}
        E_j=E\{t_j\}=\sum_{j=1}^N\varepsilon t_j
      \end{align}
      Escrevendo a função de partição de partícula única teremos
      \begin{align}
        \begin{split}
          Z_1&=\sum_t\E^{-\beta t\varepsilon}=\E^{-\beta\varepsilon_1}+\E^{-\beta\varepsilon_2}+\E^{-\beta\varepsilon_3}\\
          Z_1&=1+\E^{-\beta\varepsilon}+\E^{-3\beta\varepsilon}
        \end{split}
      \end{align}
      e para o sistema de $N$ partículas não interagentes, basta notar que $Z=Z_1^N$, logo
      \begin{align}
        \addtolength{\fboxsep}{5pt}
        \boxed{
          \begin{gathered}
            Z=\left(1+\E^{-\beta\varepsilon}+\E^{-3\beta\varepsilon}\right)^N
          \end{gathered}
        }
      \end{align}
      \item A energia interna por partícula fica
      \begin{align}
        \begin{split}
          -u&=\frac{1}{N}\parder{}{\beta}\ln Z\\
          -u&=\frac{1}{N}N\parder{}{\beta}\ln\left(1+\E^{-\beta\varepsilon}+\E^{-3\beta\varepsilon}\right)\\
          -u&=\frac{1}{1+\E^{-\beta\varepsilon}+\E^{-3\beta\varepsilon}}\left(-\varepsilon\E^{-\beta\varepsilon}-3\varepsilon\E^{-3\beta\varepsilon}\right)
        \end{split}
      \end{align}
      \begin{align}
        \addtolength{\fboxsep}{5pt}
        \boxed{
          \begin{gathered}
            u(T)=\frac{\varepsilon\left(1+3\E^{-2\varepsilon/k_BT}\right)}{1+\E^{\varepsilon/k_BT}+\E^{-2\varepsilon/k_BT}}
          \end{gathered}
        }
      \end{align}
      Para $T\to 0$ tem-se
      \begin{align}
        \begin{split}
          u(T\to 0)&=\lim_{T\to 0}\frac{\varepsilon\left(1+3\E^{-2\varepsilon/k_BT}\right)}{1+\E^{\varepsilon/k_BT}+\E^{-2\varepsilon/k_BT}}\\
          u(T\to 0)&=\varepsilon\left(\frac{1}{1+\E^{\infty}}\right)
        \end{split}
      \end{align}
      \begin{align}
        \addtolength{\fboxsep}{5pt}
        \boxed{
          \begin{gathered}
            u(T\to 0)=0
          \end{gathered}
        }
      \end{align}
      e se $T\to\infty$
      \begin{align}
        \begin{split}
          u(T\to\infty)&=\lim_{T\to\infty}\frac{\varepsilon\left(1+3\E^{-2\varepsilon/k_BT}\right)}{1+\E^{\varepsilon/k_BT}+\E^{-2\varepsilon/k_BT}}\\
          u(T\to\infty)&=\frac{\varepsilon\left(1+3(\E^{-0})\right)}{1+\E^0+\E^{-0}}
        \end{split}
      \end{align}

      \begin{align}
        \addtolength{\fboxsep}{5pt}
        \boxed{
          \begin{gathered}
            u(T\to\infty)=\frac{4\varepsilon}{3}
          \end{gathered}
        }
      \end{align}
      \item Obtem-se a entropia por partícula novamente pela função de Helmholtz $f$, sendo
      \begin{align}
        \begin{split}
          f&=-\frac{1}{\beta}\lim_{N\to\infty}\frac{1}{N}\ln Z\\
          f&=-\frac{1}{\beta}\ln\left(1+\E^{-\beta\varepsilon}+\E^{-3\beta\varepsilon}\right)
        \end{split}
      \end{align}
      logo
      \begin{align}
        \begin{split}
          -s&=\parder{f}{T}=\parder{f}{\beta}\frac{\dif\beta}{\dif T}\\
          -s&=-\parder{}{\beta}\left[\frac{1}{\beta}\ln\left(1+\E^{-\beta\varepsilon}+\E^{-3\beta\varepsilon}\right)\right]\frac{\dif\beta}{\dif T}\\
          s&=-\left[\frac{1}{\beta^2}\ln\left(1+\E^{-\beta\varepsilon}+\E^{-3\beta\varepsilon}\right)\right]\frac{\dif\beta}{\dif T}-\frac{1}{\beta}\left[\frac{\varepsilon\left(1+3\E^{-2\beta\varepsilon}\right)}{1+\E^{\beta\varepsilon}+\E^{-2\beta\varepsilon}}\right]\frac{\dif\beta}{\dif T}\\
          s(T)&=-\left(k_BT\right)^2\left[\ln\left(1+\E^{-\varepsilon/k_BT}+\E^{-3\varepsilon/k_BT}\right)\right]\left(-\frac{1}{k_BT^2}\right)+\\
          &\qquad-k_BT\left[\frac{\varepsilon\left(1+3\E^{-2\varepsilon/k_BT}\right)}{1+\E^{\varepsilon/k_BT}+\E^{-2\varepsilon/k_BT}}\right]\left(-\frac{1}{k_BT^2}\right)
        \end{split}
      \end{align}
      \begin{align}
        \label{eq:entropia-prob3}
        \addtolength{\fboxsep}{5pt}
        \boxed{
          \begin{gathered}
            s(T)=k_B\ln\left(1+\E^{-\varepsilon/k_BT}+\E^{-3\varepsilon/k_BT}\right)+\frac{1}{T}\left[\frac{\varepsilon\left(1+3\E^{-2\varepsilon/k_BT}\right)}{1+\E^{\varepsilon/k_BT}+\E^{-2\varepsilon/k_BT}}\right]
          \end{gathered}
        }
      \end{align}
      \item E pro calor específico $c$ tem-se que
      \begin{align}
        c&=T\parder{s}{T}
      \end{align}
      fazendo a substituição
      \begin{align}
        q(T)&=-\frac{\varepsilon}{k_BT}
      \end{align}
      em \eqref{eq:entropia-prob3}, podemos reescrevê-la como
      \begin{align}
        s(q)=k_B\ln\left(1+\E^{q}+\E^{3q}\right)-k_B\left[\frac{q\left(1+3\E^{2q}\right)}{1+\E^{-q}+\E^{2q}}\right]
      \end{align}
      se
      \begin{subequations}
        \begin{align}
          s(q)&=k_Bs_1(q)-k_Bs_2(q)\label{eq:sq}\\
          s_1&=\ln\left(1+\E^q+\E^{3q}\right)\label{eq:s1}\\
          s_2&=\frac{q\left(1+3\E^{2q}\right)}{1+\E^{-q}+\E^{2q}}\label{eq:s2}
        \end{align}
      \end{subequations}
      então
      \begin{align}
        \parder{s(q)}{T}&=\parder{s}{q}\frac{\dif q}{\dif T}=k_B\left[\parder{s_1}{q}\frac{\dif q}{\dif T}-\parder{s_2}{q}\frac{\dif q}{\dif T}\right]\label{eq:ds-de-T}
      \end{align}
      Calculando as derivadas
      \begin{subequations}
        \begin{align}
          \parder{s_1}{q}&=\frac{\E^q+3\E^{3q}}{1+\E^{q}+\E^{3q}}=\frac{1+3\E^{2q}}{1+\E^{-q}+\E^{2q}}\label{eq:ds1}\\
          \parder{s_2}{q}&=\frac{1+3\E^{2q}}{1+\E^{-q}+\E^{2q}}+q\frac{4\E^{2q}+9\E^{q}+\E^{-q}}{\left(1+\E^{-q}+\E^{2q}\right)^2}\label{eq:ds2}\\
          \frac{\dif q}{\dif T}&=\frac{\varepsilon}{k_BT^2}\label{eq:dq-de-T}
        \end{align}
      \end{subequations}
      substituindo as derivadas \eqref{eq:ds1} e \eqref{eq:ds2} em \eqref{eq:ds-de-T} ficamos com
      \begin{align}
        \begin{split}
          \parder{s(q)}{T}&=k_B\Biggl\{\frac{1+3\E^{2q}}{1+\E^{-q}+\E^{2q}}\left(\frac{\dif q}{\dif T}\right)-\left[\frac{1+3\E^{2q}}{1+\E^{-q}+\E^{2q}}+\right.\\
          &\qquad\left.+q\frac{4\E^{2q}+9\E^{q}+\E^{-q}}{\left(1+\E^{-q}+\E^{2q}\right)^2}\right]\left(\frac{\dif q}{\dif T}\right)\Biggr\}\\
          \parder{s(q)}{T}&=-k_B\left[q\frac{4\E^{2q}+9\E^{q}+\E^{-q}}{\left(1+\E^{-q}+\E^{2q}\right)^2}\left(\frac{\dif q}{\dif T}\right)\right]\\
          \parder{s}{T}&=-k_B\left(-\frac{\varepsilon}{k_BT}\right)\left(\frac{\varepsilon}{k_BT^2}\right)\left[\frac{4\E^{-2\varepsilon/k_BT}+9\E^{-\varepsilon/k_BT}+\E^{\varepsilon/k_BT}}{\left(1+\E^{\varepsilon/k_BT}+\E^{-2\varepsilon/k_BT}\right)^2}\right]\\
          \frac{c}{T}=\parder{s}{T}&=k_B\left(\frac{\varepsilon^2}{k_B^2T^3}\right)\left[\frac{4\E^{-2\varepsilon/k_BT}+9\E^{-\varepsilon/k_BT}+\E^{\varepsilon/k_BT}}{\left(1+\E^{\varepsilon/k_BT}+\E^{-2\varepsilon/k_BT}\right)^2}\right]
        \end{split}
      \end{align}
      \begin{align}
        \addtolength{\fboxsep}{5pt}
        \boxed{
          \begin{gathered}
            c(T)=k_B\left(\frac{\varepsilon}{k_BT}\right)^2\left[\frac{4\E^{-2\varepsilon/k_BT}+9\E^{-\varepsilon/k_BT}+\E^{\varepsilon/k_BT}}{\left(1+\E^{\varepsilon/k_BT}+\E^{-2\varepsilon/k_BT}\right)^2}\right]
          \end{gathered}
        }
      \end{align}
      Um gráfico do calor específico em função da temperatura pode ser visto a seguir
      \begin{figure}[!ht]
        \begin{center}
          \begin{tikzpicture} 
            \begin{axis}[
                axis lines = left,                
                xmin = 0, xmax = 5,
                ymin = 0, ymax = 0.5,
                xlabel = \(T\),
                ylabel = {\(c(T)\)},
                ylabel style={rotate=-90},
                ytick = {0,0.5},
                %yticklabels = {$k_B\frac{\ln 2}{2}$,$k_B\ln 2$},
                xtick = {0},
                %xticklabels = {$0$,$T_1$},
                legend pos = north east,
              ]
                \addplot[
                    domain = 0:100,
                    samples = 1000,
                    smooth,
                    thick,
                    javapurple,
                ] {((1/x)^2)*((4*exp(-2/x)+9*exp(-1/x)+exp(1/x))/(1+exp(1/x)+exp(-2/x))^2)};                
                \addlegendentry{\(c(T)\)}
                \addplot[
                    domain = 0:100,
                    samples = 1000,
                    dashed,
                    thick,
                    gray0.5,
                ] {0.682};
            \end{axis}         
          \end{tikzpicture}
        \end{center}
        \caption{Gráfico do calor específico $c(T).$}
        \label{fig:plot-prob3c}
      \end{figure}
    \end{enumerate}    
  \end{sol}
\end{prob}
\newpage
% --------------| Q04 |--------------------
\addcontentsline{toc}{section}{Problema 04}
\begin{prob}
  Um sistema de $N$ osciladores quânticos localizados e independentes está em contato com um reservatório térmico a temperatura $T$. Os níveis de energia de cada oscilador são dados por
  \begin{dmath*}
    \varepsilon=\hbar\omega_0\left(n+\frac{1}{2}\right) \condition{com $n=1,3,5,7,9...$ (ímpar)}
  \end{dmath*}
  \begin{enumerate}[label=\alph *)]
    \item Obtenha a função de partição do problema.
    \item Obtenha a expressão para a energia interna por partícula em função da temperatura $T$. Qual a expressão de $u$ no limite clássico $(\hbar\omega_0\ll k_BT)$?
    \item Qual a expressão do calor específico no limite clássico?
  \end{enumerate}
  \begin{sol}
    Considerando a temperatura do reservatório como constante $T$, podemos usar a função de partição dada pelo ensemble canônico.
    \begin{enumerate}[label=\alph *)]
      \item Seja a função de partição
      \begin{align}
        Z&=\sum_{j}\E^{-\beta E_J}
      \end{align}
      A energia total em termos dos níveis de energia de cada oscilador fica
      \begin{dmath*}
        E_J=\hbar\omega_0\sum_{r=1}^\infty\left(n_r+\frac{1}{2}\right) \condition{com $n_r=1,3,5,7,9,...$}
      \end{dmath*}
      Uma vez que os $N$ osciladores são independentes, podemos usar a função de partição de partícula única para escrever a função de partição total, segue que
      \begin{align}
        \begin{split}
          Z_1&=\sum_{n_r=1,3,5,...}^\infty\exp{\left[-\beta\hbar\omega_0\left(n_r+\frac{1}{2}\right)\right]}\\
          Z_1&=\E^{-\beta\hbar\omega_0/2}\sum_{n_r=1,3,5,...}^\infty\exp\left(-\beta\hbar\omega_0 n_r\right)
        \end{split}
      \end{align}
      O somatório acima pode ser escrito como
      \begin{align}
        \sum_{n_r=1,3,5,...}^\infty\exp\left(-\beta\hbar\omega_0 n_r\right)&=\frac{\E^{-\beta\hbar\omega_0}}{1-\E^{-2\beta\hbar\omega_0}}
      \end{align}
      \begin{proof}
        Considere a soma
        \begin{align}
          \E^{-x}+\E^{-3x}+\E^{-5x}+\E^{-7x}+...
        \end{align}
        Que pode ser reescrita pelo somatório
        \begin{align}
          \sum_{n=0}^\infty\E^{-(2n+1)x}&=\E^{-x}+\E^{-3x}+\E^{-5x}+\E^{-7x}+...
        \end{align}
        note que
        \begin{align}
          \begin{split}
            \E^{-2x}\sum_{n=0}^\infty\E^{-(2n+1)x}&=\E^{-2x}\left(\E^{-x}+\E^{-3x}+\E^{-5x}+\E^{-7x}+...\right)\\
            \E^{-2x}\sum_{n=0}^\infty\E^{-(2n+1)x}&=\E^{-3x}+\E^{-5x}+\E^{-7x}+\E^{-9x}+...
          \end{split}
        \end{align}
        mas
        \begin{align}
          \sum_{n=1}^\infty\E^{-(2n+1)}&=\E^{-3x}+\E^{-5x}+\E^{-7x}+\E^{-9x}+...
        \end{align}
        então
        \begin{align}
          \begin{split}
            \E^{-2x}\sum_{n=0}^\infty\E^{-(2n+1)x}&=\sum_{n=1}^\infty\E^{-(2n+1)}\\
            \E^{-2x}\Biggl[\sum_{n=1}^\infty\E^{-(2n+1)}+\E^{-x}\Biggr]&=\sum_{n=1}^\infty\E^{-(2n+1)}\\
            \E^{-2x}\sum_{n=1}^\infty\E^{-(2n+1)}-\sum_{n=1}^\infty\E^{-(2n+1)}&=-\E^{-3x}\\
            \left(\E^{-2x}-1\right)\sum_{n=1}^\infty\E^{-(2n+1)}&=-\E^{-3x}\\
            \sum_{n=1}^\infty\E^{-(2n+1)}&=-\frac{\E^{-3x}}{\E^{-2x}-1}\\
            \therefore
            \sum_{n=0}^\infty\E^{-(2n+1)}&=\frac{\E^{-x}}{1-\E^{-2x}}
          \end{split}
        \end{align}
      \end{proof}
      Ou seja, a função de partição de partícula única é
      \begin{align}
        \begin{split}
          Z_1&=\E^{-\beta\hbar\omega_0/2}\left[\frac{\E^{-\beta\hbar\omega_0}}{1-\E^{-2\beta\hbar\omega_0}}\right]\\
          Z_1&=\frac{\E^{-3\beta\hbar\omega_0/2}}{1-\E^{-2\beta\hbar\omega_0}}
        \end{split}
      \end{align}
      e por fim
      \begin{align}
        \addtolength{\fboxsep}{5pt}
        \boxed{
          \begin{gathered}
            Z=\left(\frac{\E^{-3\beta\hbar\omega_0/2}}{1-\E^{-2\beta\hbar\omega_0}}\right)^N
          \end{gathered}
        }        
      \end{align}
      \item Se a energia interna por partícula é dada pela expressão
      \begin{align}
        u&=-\frac{1}{N}\parder{}{\beta}\ln Z
      \end{align}
      basta aplicarmos essa definição a função de partição encontrada, portanto
      \begin{align}
        \begin{split}
          u&=-\frac{1}{N}N\parder{}{\beta}\ln Z_1\\
          u&=-\frac{1}{N}N\parder{}{\beta}\ln \left(\frac{\E^{-3\beta\hbar\omega_0/2}}{1-\E^{-2\beta\hbar\omega_0}}\right)\\
          u&=-\parder{}{\beta}\left[\ln\left(\E^{-3\beta\hbar\omega_0/2}\right)-\ln\left(1-\E^{-2\beta\hbar\omega_0}\right)\right]\\
          u&=-\parder{}{\beta}\left[-\frac{3\beta\hbar\omega_0}{2}-\ln\left(1-\E^{-2\beta\hbar\omega_0}\right)\right]\\
          u&=\frac{3\hbar\omega_0}{2}-\frac{2\hbar\omega_0 \E^{-2\beta\hbar\omega_0}}{1-\E^{-2\beta\hbar\omega_0}}\\
          u&=\frac{3\hbar\omega_0}{2}+\frac{2\hbar\omega_0}{\E^{2\beta\hbar\omega_0}-1}
        \end{split}
      \end{align}
      A expressão para a energia interna por partícula em função da temperatura $u(T)$, tem a forma
      \begin{align}
        \addtolength{\fboxsep}{5pt}
        \boxed{
          \begin{gathered}
            u(T)=\frac{3\hbar\omega_0}{2}+\frac{2\hbar\omega_0}{\E^{2\hbar\omega_0/k_BT}-1}
          \end{gathered}
        }
      \end{align}
      No limite em que $\hbar\omega_0\ll k_BT$ equivale fazer $T\to\infty$. Considere a substituição a seguir:
      \begin{dmath*}
        x\equiv \frac{\hbar\omega_0}{k_BT} \condition{se $T\to\infty$, então $x\to 0$}
      \end{dmath*}
      logo
      \begin{align}
        \begin{split}
          u&=\frac{3xk_BT}{2}+\frac{2xk_BT}{\E^{2x}-1}\\
          \frac{u}{k_BT}&=\lim_{x\to 0}\left(\frac{3x}{2}+\frac{2x}{\E^{2x}-1}\right)=0+\frac{0}{0}
        \end{split}
      \end{align}
      levantando a indeterminação do limite acima via regra de L'Hopital, teremos
      \begin{align}
        \begin{split}
          \frac{u}{k_BT}&=\lim_{x\to 0}\frac{2}{2\E^x}=1
        \end{split}
      \end{align}
      ou seja, no regime de altas temperaturas, o comportamento esperado para a energia por partícula dos $N$ osciladores quânticos, é aquele previsto pela teoria clássica.
      \begin{align}
        \addtolength{\fboxsep}{5pt}
        \boxed{
          \begin{gathered}
            u= k_BT
          \end{gathered}
        }
      \end{align}
      \item Para obter a expressão para o calor específico, faz-se necessário antes construir a energia livre de Helmholtz e a partir dela a entropia, portanto
      \begin{align}
        \begin{split}
          f&=-\frac{1}{\beta}\lim_{N\to\infty}\frac{1}{N}\ln Z\\
          f&=-\frac{1}{\beta}\ln Z_1\\
          f&=-\frac{1}{\beta}\ln\left(\frac{\E^{-3\beta\hbar\omega_0/2}}{1-\E^{-2\beta\hbar\omega_0}}\right)\\
          f&=\frac{3\hbar\omega_0}{2}+\frac{1}{\beta}\ln\left(1-\E^{-2\beta\hbar\omega_0}\right)\\
          f&=\frac{3\hbar\omega_0}{2}+k_BT\ln\left(1-\E^{-2\hbar\omega_0/k_BT}\right)
        \end{split}
      \end{align}
      calculando a entropia $s$
      \begin{align}
        \begin{split}
          s&=-\parder{f}{T}\\
          -s&=\parder{}{T}\left[\frac{3\hbar\omega_0}{2}+k_BT\ln\left(1-\E^{-2\hbar\omega_0/k_BT}\right)\right]\\
          -s&=k_B\ln\left(1-\E^{-2\hbar\omega_0/k_BT}\right)-\frac{2k_BT\hbar\omega_0}{k_BT^2}\left[\frac{\E^{-2\hbar\omega_0/k_BT}}{1-\E^{-2\hbar\omega_0/k_BT}}\right]\\
          s&=-k_B\ln\left(1-\E^{-2\hbar\omega_0/k_BT}\right)+\frac{1}{T}\left[\frac{2\hbar\omega_0}{\E^{2\hbar\omega_0/k_BT}-1}\right]
        \end{split}
      \end{align}
      De posse da expressão da entropia, podemos obter o calor específico $c$, apenas por
      \begin{align}
        \label{eq:c-prob4c}
        \begin{split}
        c&=T\parder{s}{T}\\
        \frac{c}{T}&=\parder{}{T}\left[-k_B\ln\left(1-\E^{-2\hbar\omega_0/k_BT}\right)+\frac{1}{T}\left(\frac{2\hbar\omega_0}{\E^{2\hbar\omega_0/k_BT}-1}\right)\right]\\
        \frac{c}{T}&=\Biggl\{\frac{1}{T^2}\left[\frac{(2\hbar\omega_0)\E^{-2\hbar\omega_0/k_BT}}{1-\E^{-2\hbar\omega_0/k_BT}}\right]-\frac{1}{T^2}\left[\frac{2\hbar\omega_0}{\E^{2\hbar\omega_0/k_BT}-1}\right]+\\
        &\qquad+\frac{1}{T^2}\left[\frac{4\hbar^2\omega_0^2\E^{2\hbar\omega_0/k_BT}}{k_BT\left(\E^{2\hbar\omega_0/k_BT}-1\right)^2}\right]\Biggr\}\\
        \frac{c}{T}&=\Biggl\{\frac{1}{T^2}\left[\frac{2\hbar\omega_0}{\E^{2\hbar\omega_0/k_BT}+1}\right]-\frac{1}{T^2}\left[\frac{2\hbar\omega_0}{\E^{2\hbar\omega_0/k_BT}-1}\right]+\\
        &\qquad+\frac{4\hbar^2\omega_0^2}{k_BT^3}\left[\frac{\E^{2\hbar\omega_0/k_BT}}{\left(\E^{2\hbar\omega_0/k_BT}-1\right)^2}\right]\Biggr\}\\
        c&=4k_B\left(\frac{\hbar\omega_0}{k_BT}\right)^2\frac{\E^{2\hbar\omega_0/k_BT}}{\left(\E^{2\hbar\omega_0/k_BT}-1\right)^2}
        \end{split}
      \end{align}
      Se fizermos novamente
      \begin{align}
        x\equiv \frac{\hbar\omega_0}{k_BT}
      \end{align}
      e substituirmos $x$ na \eqref{eq:c-prob4c} (note que $x\to 0$ se $T\to \infty$), ficamos com
      \begin{align}
        \label{eq:c-de-x-prob4c}
        c&=4k_B\frac{x^2\E^{2x}}{\left(\E^{2x}-1\right)^2}
      \end{align}
      manipulando a \eqref{eq:c-de-x-prob4c}, obtêm-se
      \begin{align}
        \begin{split}
          c&=4k_B\frac{\E^{2x}}{\left(\frac{\E^{2x}-1}{x}\right)\left(\frac{\E^{2x}-1}{x}\right)}
        \end{split}
      \end{align}
      fazendo a substituição $2x=t$, ($t\to 0$ se $x\to 0$)
      \begin{align}
        \begin{split}
          c&=4k_B\frac{\E^t}{2\left(\frac{\E^t-1}{t}\right)2\left(\frac{\E^t-1}{t}\right)}
        \end{split}
      \end{align}
      aplicando-se as propriedades de limites e o limite notável
      \begin{align}
        \lim_{t\to 0}\frac{\E^t-1}{t}&=1
      \end{align}
      chegamos finalmente a
      \begin{align}        
        \begin{split}
          c=4k_B\frac{1}{4}          
        \end{split}
      \end{align}
      isto é, no limite de altas temperaturas, os efeitos quânticos desaparecem, prevalecendo somente a previsão clássica para o calor específico
      \begin{align}
        \addtolength{\fboxsep}{5pt}
        \boxed{
          \begin{gathered}
            c=k_B
          \end{gathered}
        }
      \end{align}
      \textcolor {darkred} {
        Obs: Este resultado pode ser obtido diretamente tomando a derivada da energia, já calculada no limite clássico $u(T)=k_BT$
        \begin{align}
          c&=\parder{u(T)}{T}=\frac{\dif}{\dif T}\left(k_BT\right)\nonumber\\
          c&=k_B\nonumber
        \end{align}
        mas creio que não seja o objetivo desta disciplina.
      }
    \end{enumerate}
  \end{sol}
\end{prob}
% --------------| Q05 |--------------------
\addcontentsline{toc}{section}{Problema 05}
\begin{prob}
  Em aula, mostramos que a função de partição de um conjunto de $N$ osciladores clássicos em uma dimensão, definido pelo hamiltoniano
  \begin{align}
    \mathcal{H}&=\sum_{i=1}^N\left(\frac{1}{2m}p^2_i+\frac{1}{2}m\omega^2q_i^2\right)
  \end{align}
  é dada por
  \begin{align}
    Z&=\left(\frac{2\pi}{\beta\omega}\right)^N0
  \end{align}
  \begin{enumerate}[label=\alph *)]
    \item Calcule a energia por partícula.
    \item Calcule a entropia por partícula.
    \item Mostre que o calor específico é dado pela Lei de Dulong-Petit.
  \end{enumerate}
  \begin{sol}
    Dado que conhecemos $Z$, a energia por partícula $u$ é simplesmente
    \begin{enumerate}[label=\alph *)]
      \item 
      \begin{align}
        \begin{split}
          u&=-\frac{1}{N}\parder{}{\beta}\ln Z\\
          u&=-\parder{}{\beta}\ln\left(\frac{2\pi}{\beta\omega}\right)\\
          u&=-\frac{\beta\omega}{2\pi}\left(-\frac{2\pi}{\beta^2\omega}\right)\\
          u&=\frac{1}{\beta}
        \end{split}
      \end{align}
      \begin{align}
        \addtolength{\fboxsep}{5pt}
        \boxed{
          \begin{gathered}
            u(T)=k_BT
          \end{gathered}
        }
      \end{align}
      \item Vamos obter a entropia pela energia livre de Helmholtz $f$
      \begin{align}
        \begin{split}
          f&=-\frac{1}{\beta}\lim_{N\to \infty}\frac{1}{N}\ln Z\\
          f&=-k_BT\ln\left(\frac{2\pi k_BT}{\omega}\right)
        \end{split}
      \end{align}
      mas
      \begin{align}
        \begin{split}
          -s&=\parder{f}{T}\\
          -s&=-k_B\ln\left(\frac{2\pi k_BT}{\omega}\right)-k_BT\left[\frac{\omega}{2\pi k_BT}\left(\frac{2\pi k_B}{\omega}\right)\right]          
        \end{split}
      \end{align}
      logo
      \begin{align}
        \addtolength{\fboxsep}{5pt}
        \boxed{
          \begin{gathered}
            s=k_B\left[\ln\left(\frac{2\pi k_BT}{\omega}\right)+1\right]
          \end{gathered}
        }
      \end{align}
      \item Determinando o calor específico
      \begin{align}
        \begin{split}
          c&=T\parder{s}{T}\\
          \frac{c}{T}&=k_B\parder{}{T}\left[\ln\left(\frac{2\pi k_BT}{\omega}\right)+1\right]\\
          \frac{c}{T}&=k_B\frac{\omega}{2\pi k_BT}\left(\frac{2\pi k_B}{\omega}\right)\\
          \frac{c}{T}&=\frac{k_B}{T}
        \end{split}
      \end{align}
      portanto
      \begin{align}
        \addtolength{\fboxsep}{5pt}
        \boxed{
          \begin{gathered}
            c=k_B
          \end{gathered}
        }
      \end{align}
    \end{enumerate}
  \end{sol}
\end{prob}
% --------------| Q06 |--------------------
\addcontentsline{toc}{section}{Problema 06}
\begin{prob}
  Um sistema de $N$ partículas clássicas ultra-relativísticas, dentro de um recipiente de volume $V$, a uma temperatura $T$, é definido pelo hamiltoniano
  \begin{align}
    \mathcal{H}&=\sum_{i=1}^Nc|\vec{p}_i|
  \end{align}
  onde $c$ é uma constante positiva (velocidade da luz).
  \begin{enumerate}[label=\alph *)]
    \item Obtenha uma expressão para a função canônica de partição. \textbf{Sugestão}: escreva o elemento de volume nos momenta ($\dif^3p$) em coordenadas esféricas.
    \item Obtenha a entropia por partícula, como função da temperatura e do volume específico.
    \item Qual a expressão do calor específico a volume constante.
  \end{enumerate}
  \begin{sol}
    A função de partição clássica deste sistema, é dada pela integral no espaço de fase
    \begin{align}
      Z_c&=\idotsint\limits_{V} \dif^3\vec{r}_1 \dots \dif^3\vec{r}_N \idotsint \dif^3\vec{p}_1 \dots \dif^3\vec{p}_N\,\exp\left(-\beta\mathcal{H}\right)
    \end{align}
    cujo os fatores de correção, que a tornam adimensional ($h$) e eliminam o paradoxo de de Gibbs ($1/N!$), são introduzidos gerando a função de partição canônica \emph{(corrigida)}
    \begin{align}
      Z&=\frac{1}{N!}\frac{1}{h^{3N}}Z_c
    \end{align}
    \begin{enumerate}[label=\alph *)]
      \item Considerando o sistema como não interagente ($N$ partículas ultra relativísticas), de modo que podemos analisar a função de partição para uma única partícula e depois estender a análise para as demais, tem-se que
      \begin{dmath*}
        Z_{1c}=\int \dif^3{\vec{r}_1}  \int \E^{-\beta\mathcal{H}_1} \dif^3{\vec{p}_1} \condition{com $\mathcal{H}_1=c|\vec{p}_1|$}
      \end{dmath*}
      dado que o módulo dos momentos, é simplesmente
      \begin{align}
        |\vec{p}_1|&=\sqrt{p_{1x_1}^2+p_{1x_2}^2+p_{1x_3}^2}
      \end{align}
      adotando a sugestão do problema e escrevendo os momentos e os elementos diferenciais dos momentos em coordenadas esféricas
      \begin{subequations}
        \begin{align}
          |\vec{p}_1|&=r\\
          \dif^3{\vec{p}_1}&=r^2\sen\theta\dif{r}\dif{\theta} \dif{\varphi}
        \end{align}
      \end{subequations}
      atualizamos para
      \begin{dmath*}
        Z_{1c}=V\iiint \E^{-c\beta r} r^2\dif{r} \dif{\theta} \dif{\varphi} \condition{com $V=\displaystyle\int\limits_{V}\dif^3{r_1}$}
      \end{dmath*}
      As integrais nas variáveis angulares são triviais, resultando em $4\pi$. Computando a integral em $r$, com a substituição $s=rc\beta$ se $r\to 0$, $s\to 0$ e se $r\to \infty$ então $s\to \infty$, portanto
      \begin{align}
        \begin{split}
          \int\limits_0^\infty r^2\E^{-rc\beta} \dif{r}&=\int\limits_0^\infty \frac{s^2}{\beta^2c^2}\E^{-s} \frac{1}{\beta c} \dif{s}\\
          &=\frac{1}{c^3\beta^3}\int\limits_0^\infty s^2\E^{-s} \dif{s}\\
          &=-\frac{2}{c^3\beta^3}\Biggl[\CancelTo[\color{deepgreen}]{-1}{\frac{1}{\E^{s}}}\Biggr |_{0}^{\infty}+\CancelTo[\color{deepred}]{0}{\frac{s}{\E^{s}}\Biggr |_{0}^{\infty}}+\CancelTo[\color{deepred}]{0}{\frac{s^2}{2\E^{s}}\Biggr |_{0}^{\infty}}\Biggr]=\frac{2}{c^3\beta^3}
        \end{split}
      \end{align}
      ou seja
      \begin{align}
        Z_{1c}&=4\pi V\frac{2}{c^3\beta^3}
      \end{align}
      Para as $N$ partículas devemos ter $Z_c=Z_{1c}^N$ o que nos da para a função de partição canônica $Z$ o seguinte resultado
      \begin{align}
        \addtolength{\fboxsep}{5pt}
        \boxed{
          \begin{gathered}
            Z=\frac{V^N}{N!}\left(\frac{2\sqrt[3]{\pi}}{hc\beta}\right)^{3N}
          \end{gathered}
        }      
      \end{align}
      \item Para obter a entropia por partícula, precisamos antes contruir a função energia livre de Helmholtz por partícula $f$ e estabelecer a conexão com a termodinâmica
      \begin{align}
        f(T,v)=-\frac{1}{\beta}\lim_{\substack{V,N\to\infty\\\frac{V}{N}\to v}}\frac{1}{N}\ln Z
      \end{align}
      sendo assim, temos que
      \begin{align}
        f(T,v)&=-\frac{1}{\beta}\lim_{\substack{V,N\to\infty\\\frac{V}{N}\to v}}\frac{1}{N}\ln\left[\frac{V^N}{N!}\left(\frac{2\sqrt[3]{\pi}}{hc\beta}\right)^{3N}\right]\nonumber
      \end{align}
      \begin{align}
        \begin{split}          
          f(T,v)&=-\frac{1}{\beta}\lim_{\substack{V,N\to\infty\\\frac{V}{N}\to v}}\frac{1}{N}\Biggl\{N
          \textcolor{deepblue}{%
            \ln V
          }-\textcolor{deepred}{%
            \ln N!
          }+3N\left[\ln\left(\frac{2\sqrt[3]{\pi}}{hc}\right)-\ln\beta\right]\Biggr\}\\
          f(T,v)&=-\frac{1}{\beta}\lim_{\substack{V,N\to\infty\\\frac{V}{N}\to v}}\frac{1}{N}\left[N
          \textcolor{deepblue}{%
            \left(\ln N+\ln \frac{V}{N}\right)
          }-\textcolor{deepred}{%
            \left(N\ln N-N\right)
          }+\right.\\
          &\qquad\left.+3N\ln\left(\frac{2\sqrt[3]{\pi}}{hc}\right)-3N\ln\beta\right]\\
          f(T,v)&=-\frac{1}{\beta}\lim_{\substack{V,N\to\infty\\\frac{V}{N}\to v}}\frac{1}{N}\left[N\ln N +N\ln\left(\frac{V}{N}\right)-N\ln N +N+\right.\\&\qquad\left.-3N\ln \beta+3N\ln\left(\frac{2\sqrt[3]{\pi}}{hc}\right)\right]\\
          f(T,v)&=-\frac{1}{\beta}\lim_{\substack{V,N\to\infty\\\frac{V}{N}\to v}}\left[1+\ln\left(\frac{V}{N}\right)-3\ln\beta+3\ln\left(\frac{2\sqrt[3]{\pi}}{hc}\right)\right]\\
          f(T,v)&=-k_BT\left[1+\ln v-3\ln\left(\frac{1}{k_BT}\right)+3\ln\left(\frac{2\sqrt[3]{\pi}}{hc}\right)\right]\\
          f(T,v)&=-k_BT-k_BT\ln v-3k_BT\ln\left(k_BT\right)-3k_BT\ln\left(\frac{2\sqrt[3]{\pi}}{hc}\right)
        \end{split}
      \end{align}
      Logo, a entropia por partícula fica
      \begin{align}
        \begin{split}
          -s&=\left(\parder{f}{T}\right)_v\\
          s&=\parder{}{T}\left[k_BT+k_BT\ln v+3k_BT\ln\left(k_BT\right)+3k_BT\ln\left(\frac{2\sqrt[3]{\pi}}{hc}\right)\right]\\
          s&=k_B+k_B\ln v+3k_B\ln\left(k_BT\right)+3k_BT\left(\frac{1}{k_BT}\right)k_B+3k_B\ln\left(\frac{2\sqrt[3]{\pi}}{hc}\right)
        \end{split}
      \end{align}
      \begin{align}
        \addtolength{\fboxsep}{5pt}
        \boxed{
          \begin{gathered}
            s=4k_B+k_B\ln v+3k_B\ln\left(\frac{2k_BT\sqrt[3]{\pi}}{hc}\right)
          \end{gathered}
        }
      \end{align}
      \item O calor específico $c_V$, fica
      \begin{align}
        \begin{split}
          c_V&=T\left(\parder{s}{T}\right)_V\\
          \frac{c_V}{T}&=3k_B\left(\frac{hc}{2k_BT\sqrt[3]{\pi}}\right)\left(\frac{2k_B\sqrt[3]{\pi}}{hc}\right)\\
          \frac{c_V}{T}&=\frac{3k_B}{T}
        \end{split}
      \end{align}
      \begin{align}
        \addtolength{\fboxsep}{5pt}
        \boxed{
          \begin{gathered}
            c_V=3k_B
          \end{gathered}
        }
      \end{align}
    \end{enumerate}
  \end{sol}
\end{prob}
%--------------| Q07 |--------------------
\addcontentsline{toc}{section}{Problema 07}
\begin{prob}
  Considere um conjunto de N osciladores unidimensionais, descrito pelo hamiltoniano
  \begin{align}
    \mathcal{H}&=\sum_{i=1}^N\left[\frac{1}{2m}\vec{p_i}^2+\frac{1}{2}m\omega^2x^n_i\right]
  \end{align}
  onde $n$ é um número par e positivo. Utilize o formalismo canônico para obter uma expressão para o calor específico clássico do sistema em termos de $n$. Mostre que, para $n=2$, a Lei de Dulong-Petit é recuperada.
  \begin{sol}
    A função de partição canônica unidimensional é dada por 
    \begin{align}
      Z&=\idotsint\limits_{V}\dif{x_1^n}\dots\dif{x_N^n}\int \dots \int \dif{\vec{p}_1}\dots\dif{\vec{p}_N}\exp\left(-\beta\mathcal{H}\right)
    \end{align}
    Separando as coordenadas do momento da parte configuracional, ficamos com
    \begin{align}
      Z&=Q_N\int \dots \int \dif{\vec{p}_1}\dots\dif{\vec{p}_N}\exp\left[-\sum_{i=1}^N\frac{\beta\vec{p^2}_i}{2m}\right]
    \end{align}
    em que $Q_N$ é
    \begin{align}
      Q_N&=\idotsint\limits_{V}\dif{x_1^n}\dots\dif{x_N^n}\exp\left[-\sum_{j=1}^N\frac{\beta m\omega^2}{2}x_j^n\right]
    \end{align}
    \begin{enumerate}[label=\alph *)]
      \item As integrais dos momentos são gaussianas, cujo o resultado de uma delas é
      \begin{align}
        \int\limits_{-\infty}^\infty \exp\left(-\frac{\beta}{2m}p^2\right) \dif{p}&=\sqrt{\frac{2m\pi}{\beta}}
      \end{align}
      Para a parte configuracional, tem-se 
      \begin{align}
        \label{eq:qn-prob7}
        Q_N&=\prod_{j=1}^N\Biggl[\,\int\limits_{V} \exp\left(-\frac{m\omega^2\beta}{2}x_j^n\right) \dif{x_j}\Biggr]
      \end{align}
      se $j=1$ e $\alpha=m\omega^2\beta/2$, a equação \eqref{eq:qn-prob7} fica
      \begin{align}
        \label{eq:int1-prob7}
        Q_1&=\int\limits_{V}\E^{-\alpha x_1^n}\dif{x_1}
      \end{align}
      $Q_1$ não tem solução analítica para qualquer valor de $n$, tampouco é garantido a convergência da integral em todo o seu intervalo, no entanto precisamos encontrar uma expressão para o calor específico em termos de $n$, a estratégia que utilizamos é escrever estas integrais como funções $\Gamma(n)$, para tanto, utilizando a substituição $\alpha^{1/n} x_1\equiv t^{1/n}$, com $t\in [0,+\infty)$ obtendo
      \begin{align}
        \dif{x_1}=\frac{1}{n\alpha^{1/n}}t^{\frac{1}{n}-1}\dif{t}
      \end{align}
      o que reduz a integral \eqref{eq:int1-prob7} a
      \begin{align}
        \int\limits_{V}\E^{-\alpha x_1^n}\dif{x_1}&=\frac{1}{n\alpha^{1/n}}\int_{0}^\infty t^{\frac{1}{n}-1}\E^{-t}\dif{t}=\alpha^{-1/n}\frac{1}{n}\Gamma\left(\frac{1}{n}\right)
      \end{align}
      e assim, escrevemos $Q_1$ em termos de $n$
      \begin{align}
        \begin{split}
          Q_1(n)&=\left(\frac{2}{m\omega^2\beta}\right)^{1/n}\frac{1}{n}\Gamma\left(\frac{1}{n}\right)
        \end{split}
      \end{align}
      consequentemente a função de partição fica     
      \begin{align}
        \addtolength{\fboxsep}{5pt}
        \boxed{
          \begin{gathered}
            Z=\Biggl[\sqrt{\frac{2m\pi}{\beta}}\left(\frac{2}{m\omega^2\beta}\right)^{1/n}\frac{1}{n}\Gamma\left(\frac{1}{n}\right)\Biggr]^N
          \end{gathered}
        }
      \end{align}
      Se a energia interna por partícula é
      \begin{align}
        u=-\frac{1}{N}\parder{}{\beta}\ln Z
      \end{align}
      têm-se
      \begin{align}
        \begin{split}
          u&=-\frac{1}{N}\parder{}{\beta}\ln \Biggl[\sqrt{\frac{2m\pi}{\beta}}\left(\frac{2}{m\omega^2\beta}\right)^{1/n}\frac{1}{n}\Gamma\left(\frac{1}{n}\right)\Biggr]^N\\
          u&=-\parder{}{\beta}\Biggl\{\frac{1}{2}\ln\left(\frac{2m\pi}{\beta}\right)+\frac{1}{n}\ln\left(\frac{2}{m\omega^2\beta}\right)+\ln\left[\frac{1}{n}\Gamma\left(\frac{1}{n}\right)\right]\Biggr\}\\
          u&=-\left[\frac{1}{2}\left(\frac{\beta}{2m\pi}\right)\left(-\frac{2m\pi}{\beta^2}\right)+\frac{1}{n}\left(\frac{m\omega^2\beta}{2}\right)\left(-\frac{2}{m\omega^2\beta^2}\right)\right]\\
          u&=\frac{1}{2\beta}+\frac{1}{n\beta}=\left(\frac{n+2}{2n}\right)k_BT
        \end{split}
      \end{align}
      Logo o calor específico em termos de $n$ é dado simplesmente por
      \begin{align}
        c&=\parder{u}{T}
      \end{align}
      \begin{align}
        \addtolength{\fboxsep}{5pt}
        \boxed{
          \begin{gathered}
            c(n)=\left(\frac{n+2}{2n}\right)k_B
          \end{gathered}
        }
      \end{align}
      Evidentemente, quando $n=2$
      \begin{align}
        \begin{split}
          c(2)&=\frac{4}{4}k_B=k_B
        \end{split}
      \end{align} 
    \end{enumerate}
  \end{sol}
\end{prob}
%--------------| Q08 |--------------------
\addcontentsline{toc}{section}{Problema 08}
\begin{prob}[\textbf{Exercício 7.8 do Greiner}]
  Considere uma coluna de ar acima da superfície da Terra com área da base $A$. Calcule a distribuição de densidade de partículas na coluna sob influência da gravidade, numa dada temperatura $T$. Assuma que o ar se comporta como um gás ideal e assuma a gravidade constante.

  \noindent Nesta referência, usa-se a notação de comprimento de onda térmico:
  \begin{align}
    \lambda&=\left(\frac{h^2}{2\pi mk_BT}\right)^{1/2}
  \end{align}
  \begin{sol}
    O sistema formado pelas $N$ partículas componentes do ar na região delimitada, está sujeito ao potencial gravitacional $V(z)=mgz$, a energia total do sistema é dada pela soma da parte translacional $T$ com a parte potencial $V(z)$
    \begin{dmath*}
      E=T+V(z)
    \end{dmath*}
    Este sistema é conservativo, de modo tal que, o hamiltoniano deste sistema é a energia total do sistema portanto, para as $N$ partículas devemos ter
    \begin{align}
      T&=\sum_{i=1}^N\frac{1}{2}m(\dot{x}_i^2+\dot{y_i}^2+\dot{z_i}^2)\\
      V(Z_i)&=\sum_{i=1}^Nmgz_i
    \end{align}
    em termos dos momentos conjugados $\dot{\vec{p_i}}$ o hamiltoniano do sistema é simplesmente
    \begin{align}
      \mathcal{H}&=\sum_{i=1}^N\left[\frac{1}{2m}\vec{p_i}^2+mgz_i\right]
    \end{align}
    A função de partição do sistema é dada por
    \begin{align}
      Z&=\frac{1}{h^NN!}\idotsint\limits_{V}\dif^3{r_1}\dots\dif^3{r_N}\int \dots \int \dif^3{\vec{p}_1}\dots\dif^3{\vec{p}_N}\exp\left(-\beta\mathcal{H}\right)
    \end{align}
    é possível separar a parte translacional da parte configuracional (as partículas são não interagente), a parte configuracional pode ser escrita como
    \begin{align}
      Q_N&=\idotsint\limits_{V}\dif^3{\vec{r}_1}\dots\dif^3{\vec{r}_N}\exp\left(-\beta\sum_{i=1}^Nmgz_i\right)
    \end{align}
    o que para uma única partícula teremos
    \begin{dmath*}
      Q_1=\iiint\limits_{V}\E^{-\beta mgz}\dif{x}\dif{y}\dif{z}=\frac{A}{mg\beta}\condition{onde: $A=\displaystyle\int_0^{l_x}\dif{x}\int_0^{l_y}\dif{y}$}
    \end{dmath*}
    similarmente, as integrais dos momentos retornam
    \begin{dmath*}
      \int_{-\infty}^\infty\dif^3{p}\exp\left(-\frac{\beta p^2}{2m}\right)=\left(\frac{2\pi m}{\beta}\right)^{3/2}
    \end{dmath*}
    logo a função de partição associada às $N$ partículas do sistema é
    \begin{align}
      \begin{split}
        Z&=\frac{1}{h^NN!}\left(\frac{2\pi m}{\beta}\right)^{3N/2}Q_N\\
        Z&=\frac{1}{N!}\left(\frac{2\pi m}{h^2\beta}\right)^{3N/2}\left(\frac{A}{mg\beta}\right)^N
      \end{split}
    \end{align} 
  \end{sol}
\end{prob}
% --------------| Q09 |--------------------
\addcontentsline{toc}{section}{Problema 09}
\begin{prob}
  Considere um gás clássico de $N$ moléculas fracamente interagentes, a temperatura $T$, na presenção de um campo elétrico. Como não há momento de dipolo permanente, qualquer polarização será induzida pelo campo. Podemos, então, supor que o hamiltoniano de cada molécula seja dado pela soma de um termo de translação com um “termo interno”. Esse termo interno envolve uma energia elástica, isotrópica, que tende a preservar a forma da molécula, e um termo de interação com o campo. A parte configuracional do hamiltoniano interno é dada por
  \begin{align}
    \mathcal{H}&=\frac{1}{2}m\omega^2_0r^2-q\vec{E}\cdot \vec{r}
  \end{align}
  \begin{enumerate}[label=\alph *)]
    \item Calcule a função de partição de partícula única (dica: usar como elemento de volume $d^3\vec{r}=r^2dr\sin\theta d\theta d\varphi$).
    \item Calcule a polarização por molécula em função do campo e da temperatura, assim como a suscetibilidade elétrica (derivada da polarização em relação ao campo). A polarização é definida por
    \begin{align}
      \langle qr\cos\theta\rangle
    \end{align}
  \end{enumerate}
\end{prob}

% ============| FIM |======================


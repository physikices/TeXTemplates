%%%%%%%%%%%%%% Como usar o pacote acronym
% \ac{acronimo} -- Na primeira vez que for citado o acronimo, o nome completo irá aparecer
%                  seguido do acronimo entre parênteses. Na proxima vez somente o acronimo
%                  irá aparecer. Se usou a opção footnote no pacote, entao o nome por extenso
%                  irá aparecer aparecer no rodapé
%
% \acf{acronimo} -- Para aparecer com nome completo + acronimo
% \acs{acronimo} -- Para aparecer somente o acronimo
% \acl{acronimo} -- Nome por extenso somente, sem o acronimo
% \acp{acronimo} -- igual o \ac mas deixando no plural com S (ingles)
% \acfp{acronimo}--
% \acsp{acronimo}--
% \aclp{acronimo}--


\chapter*{Lista de abreviaturas e siglas}%
% \addcontentsline{toc}{chapter}{Lista de abreviaturas e siglas}
\markboth{Lista de abreviaturas e siglas}{}

% Para diminuir o espaçamento entre linhas no ambiente de listas acronym
% \let\oldbaselinestretch=\baselinestretch%
% \renewcommand{\baselinestretch}{.2}%
% \large\normalsize%



\begin{acronym}
	\let\oldbaselinestretch=\baselinestretch%
	\renewcommand{\baselinestretch}{.2}%
	\large\normalsize%
	\acro{BNCC}{Base Nacional Comum Curricular}
	\acro{LDB}{Lei de Diretrizes de Bases da Educação Nacional}
	\acro{NEM}{Novo Ensino Médio}
	\acro{PCSC}{Proposta Curricular de Santa Catarina}
	\acro{PNE}{Plano Nacional de Educação}
	\acro{CNE}{Conselho Nacional de Educação}
	\acro{EEB}{Escola de Educação Básica}
	\acro{GPF}{Giovani Pasqualini faraco}
	\acro{EF}{Ensino Fundamental}
	\acro{PPP}{Projeto Político Pedagógico}
	\acro{ATP}{Assistente Técnico Pedagógico}
	\acro{DVD}{\textit{Digital Versatile Disc}}
	\acro{EI}{Ensino por Investigação}
	\acro{NI/D}{Não-Interativo/Dialógico}
	\acro{I/D}{Interativo-Dialógico}
	\acro{HC}{História da Ciência}
	\acro{TIC}{Tecnologia da Informação e Comunicação}
	\acro{ENEM}{Exame Nacional do Ensino Médio}
	\acro{I/DA}{Interativo/Dialógico de Autoridade}
\end{acronym}

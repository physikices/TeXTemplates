%============| INÍCIO |=======================================>
\chapter{Introdução}
\label{cap:Introducao}

Desde o seu surgimento no século XI o termo estágio \emph{(do lat.: stagium)} tem sido associado à aprendizagem posta em prática num local adequado e sob supervisão \cite{COLOMBO:2014}. Durante a trajetória acadêmica, os saberes adquirido ao longo dos anos de formação são postos em prática buscando estabelecer vínculos entre o saber e o saber fazer, acompanhado por um profissional da área que orientará e corrigirá as ações desenvolvidas pelo estagiário e assim, evitar falhas no exercício de suas atribuições no momento em que estiver apto a desenvolvê-las.

Neste sentido os estágios representam para o estagiário uma oportunidade de colocar em prática os conhecimentos construídos pelo futuro profissional ao longo de todo o processo formativo, esta prática é tanto mais proveitosa quando proporcionada à situações concretas e próximas da realidade profissional. Embora nestes termos não esteja explicitado, vale ressaltar que a prática do estágio jamais deve ser confundida com aplicação de mão de obra barata, a Lei de n$^\circ$ 11.788, de 25 de setembro de 2008 determina que o estágio seja vinculado puramente ao processo educativo. O conhecimento da norma e a gestão correta do estágio pelas instituições devem ser suficientes para evitar que esta prática se difunda na forma de precarização das relações do trabalho.

\section{Documentos Norteadores da Educação Nacional}
Com a homologação da 3ª versão da \ac{BNCC} em dezembro de 2017, passa então a valer em todo o território nacional, em caráter compulsório e de forma prevista pela \ac{LDB} assim como no \ac{PNE}, as políticas educacionais voltadas a orientar a elaboração: dos currículos locais, da formação inicial e continuada dos professores, do material didático, da avaliação e do apoio pedagógico aos alunos \cite{BRASIL:2017}, a fim de assegurar e promover os direitos de aprendizagem essenciais aos educandos com vistas à formação humana integral e à construção de uma sociedade justa, democrática e inclusiva.

O texto tem como foco o desenvolvimento de \emph{competências} por meio das quais o educando, ao longo de todo o processo formativo, deva ser capaz de:

\begin{citacao}
    ``\ldots aprender a aprender, saber lidar com a informação cada vez mais disponível, atuar com discernimento e responsabilidade nos contextos das culturas digitais, aplicar conhecimentos para resolver problemas, ter autonomia para tomar decisões, ser proativo para identificar os dados de uma situação e buscar soluções, conviver e aprender com as diferenças e as diversidades.'' 
    \Ibidem[p. ~14]{BRASIL:2017}   
\end{citacao}
Excepcionalmente no ano de escrita deste relatório, as escolas de Santa Catarina iniciaram o processo de implementação da \ac{BNCC} no então denominado, \ac{NEM}. Vem então direcionando esforços para adequar a medida primeiramente a todas as turmas de primeiro ano, e tendo por pretensões concluir o processo até o ano de 2024, ano em que deve-se incluir as turmas de terceiro ano.

Não obstante, Santa Catarina também possui a sua proposta para a educação básica, aliás o Estado é pioneiro na definição de diretrizes curriculares, e desde 1988 vem elaborando e (re)elaborando a sua proposta em participação conjunta com diversos profissionais dos segmentos educativos. A \ac{PCSC} em sua versão mais recente, cita três elementos norteadores que orientaram a consolidação do modelo atual, como sendo a:

\begin{citacao}
    ``1) perspectiva de formação integral, referenciada numa concepção multidimensional de sujeito; 2) concepção de percurso formativo visando superar o etapismo escolar e a razão fragmentária que ainda predomina na organização curricular e 3) atenção à concepção de diversidade no reconhecimento das diferentes configurações identitárias e de novas modalidades de educação \cite[p.~20]{PCSC:2014}''
\end{citacao}
Em atenção a estes elementos, a proposta ainda orienta a formação dos curriculos e do \ac{PPP} no sentido de promover a:

\begin{citacao}
    ``Superação do etapismo no percurso formativo; promoção do diálogo entre as diferentes áreas do conhecimento, sem deixar de considerar as especificidades das áreas e dos componentes curriculares; escolhas teórico-metodológicas, de conhecimentos e de experiências significativas para compor o percurso formativo e que mobilizem os sujeitos para a aprendizagem; reconhecimento da diversidade de identidades e de saberes como condição político-pedagógica para o desenvolvimento da Educação Básica; ampliação de espaços de autonomia intelectual e política dos sujeitos envolvidos nos percursos formativos; exploração das interfaces entre os saberes, dos \emph{entre-lugares (sic)}, das redes, das coletividades como \emph{lócus} geradores de conhecimento; democratização da gestão dos processos educativos pela valorização e fortalecimento do trabalho coletivo \opcit[p.~27]{PCSC:2014}''
\end{citacao}

Assim, estabelece uma concepção de currículo \emph{``\ldots mutável, ou seja, ações pedagógicas que propiciem ao sujeito ser ativo em situações de pesquisa, referente ao objeto de estudo''}, destacam \cite[p.~409]{COMIOTTO:2021}. Como podemos observar, ambas as propostas atuam de forma complementares sendo ainda concordantes em muitos aspectos. Tanto a \ac{BNCC} quanto a \ac{PCSC} formam as bases legais e norteadoras da educação catarinense e o seu conhecimento é de fundamental importância para o exercício do profissional de educação que irá atuar no Estado.

Face a isso, os cursos de licenciaturas do país tem buscado promover nos currículos de graduação, o conjunto de ações adequadas a atender às exigências dos documentos norteadores. Neste processo, encontram-se as disciplinas de Estágio Curricular Supervisionado I/II/III e IV, responsáveis por oportunizar uma primeira aproximação do acadêmico com a carreira docente em ambiente escolar supervisionado, sendo um componente curricular obrigatório e indispensável nos cursos de licenciatura, de acordo com a resolução \cite{BRASIL:2002a} homologada pelo \ac{CNE} na forma do parecer de nº  CNE/CP nº 1, de 18 de Fevereiro de 2002.

\section{Referenciais Teórico-Metodológico}

Longe de reduzir a ação dos docentes a meros agentes tecnicistas, limitados a cumprir passivamente o que lhes ditam verticalmente, vê-se nos textos das Bases, uma proximidade com as concepções da filosofia \emph{deweyana}\footnote{John Dewey (1859-1952), filósofo americano que influenciou educadores de várias partes do mundo e que no Brasil inspirou o \emph{Movimento da Escola Nova}, liderado por Anísio Teixeira.}, e neste sentido, considera-se o movimento da \emph{Prática Reflexiva} proposta por \cite{ZEICHNER:1993} como elemento catalisador do pensar e repensar frequentemente a prática pedagógica, para o autor:

\begin{citacao}
``O conceito de professor como prático reflexivo reconhece a riqueza da experiência que reside na prática dos bons professores. Na perspectiva de cada professor, significa que o processo de compreensão e melhoria do seu ensino deve começar pela reflexão sobre a sua própria experiência e que o tipo de saber inteiramente tirado da experiência dos outros (mesmo de outros professores) é, no melhor dos casos, pobre e, no pior, ilusão.'' \opcit[p. ~17]{ZEICHNER:1993}
\end{citacao}
Não se trata aqui de tornar o estagiário durante o exercício do estágio, um crítico contumaz à pratica docente observada em sala de aula, mas sim de fazê-lo

\begin{citacao}
``[...]detectar e superar uma visão simplista dos problemas de ensino e aprendizagem, proporcionando dados significativos do cotidiano escolar que possibilitem uma \textbf{reflexão crítica} do trabalho a ser desenvolvido como professor e dos processos de ensino e aprendizagem em relação ao seu conteúdo específico.'' \cite[p. 11, \textbf{grifos meus}]{CARVALHO:2012} 
\end{citacao}
Assim sendo, as problematizações trazidas à tona neste trabalho, só tem sentido se vistas no âmbito de elucidar a complexa relação existente entre o ato de ensinar e a aprendizagem significativa desejada, à luz destes referenciais.

\section{Contexto}
Este estágio foi desenvolvido ao longo do segundo semestre do ano de 2022, na \ac{EEB} \ac{GPF}, para a disciplina de \disciplina \; do curso de Licenciatura em Física, onde o estagiário é convidado a desenvolver atividades relacionadas à caracterização do ambiente escolar, acompanhamentos de aulas além de proposição e execução de atividades imersivas.
 
O restante desse trabalho está organizado da seguinte maneira: no \autoref{cap: aprConcedente} é apresentada a unidade concedente do estágio, suas características estruturais e organizacionais; o \autoref{cap: apoioDocencia} é destinado a apresentação dos programas de apoio à docência; no \autoref{cap: prgFisica} faremos a apresentação dos programas da disciplina de Física; no 

% \autoref{cap: acmpDeAulas} apresentaremos o acompanhamento das aulas assistidas; as intervenções feitas em sala de aula encontram-se no \autoref{cap: docCompartilhada} e por fim, as considerações finais serão apresentados no  \autoref{cap: consideracoesFinais}.

%=============| FIM |=========================================>

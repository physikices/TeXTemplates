%-----------------------------------------------%
% Modelo de Plano de Aula com três momentos pedagógicos
%
% Autor: Rodrigo Nascimento (2022-08-12)
%-----------------------------------------------%
\begin{center}
    \begin{minipage}[!]{\linewidth}
        \begin{minipage}[!]{.19\linewidth}
            \includegraphics[width=\linewidth]{img/logo.png}           
        \end{minipage}
        \begin{minipage}[!]{.8\linewidth}
            \center
            \ABNTEXchapterfont\normalsize\MakeUppercase{\imprimirinstituicao}
            \par
            \vspace*{10pt}                     
            \ABNTEXchapterfont\normalsize\MakeUppercase{\centro}
            \par
            \vspace*{10pt}           
            \ABNTEXchapterfont\normalsize\MakeUppercase{\disciplina}
        \end{minipage}        
    \end{minipage}
    \\ \vspace{0.5cm}
    \rule{\textwidth}{.5pt}   
\end{center}

\textual
    \begin{center}
        %\textbf{Plano de Aula: Intervenção Pedagógica nº 00X}
        \section{Primeiro plano de aula}
        \par
    \end{center}        
        \noindent \textbf{Estagiário(a): }\imprimirautor 
        
        \noindent \textbf{U.E.: }EEB NOME DA ESCOLA
        
        \noindent \textbf{Série: }Xº Ano\hfill{}\textbf{Turma: }Xº--N
        
        \noindent \textbf{Aula:} 00X\hfill{}\textbf{Data:} XX/XX/2022\hfill{}\textbf{Duração:} $XX\min$
        \rule{\textwidth}{.5pt}
        \bigskip{}  
        
        %\section{Plano-01: Primeiro plano de aula}
        \noindent \begin{center}
        \textbf{Título: Título da Aula}
        \par\end{center}

        \noindent \textbf{Resumo da aula: }

        \par\noindent \textbf{Habilidades BNCC: }EM13CNT101; EM13CNT301.
        \vfill
        \subsection*{Objetivo de Aprendizagem}
        \begin{itemize}
            \item Perceber
            \item Perceber
            \item Perceber
            \item Parecer
            \item Coisarada 
        \end{itemize}
        
        \medskip{}
        \vfill
        \noindent \textbf{Dimensão Conceitual:} \emph{Dimensão 01; Dimensão 02; Dimensão 03.}
        
        
        \subsection*{Procedimento Didático} 
        \noindent \emph{1º Momento:} Título do primeiro momento.
        \par\noindent\rule{.3\textwidth}{.5pt}  
        \par\noindent \textbf{Tempo previsto:} XX minutos

        \noindent \textbf{Dinâmica:} Descrever a dinâmica do primeiro momento.

        \vspace{50pt}
        \noindent \emph{2º Momento:} Título do segundo momento.
        \par\noindent\rule{.3\textwidth}{.5pt}    
        \par\noindent \textbf{Tempo previsto: }XX minutos
        

        \noindent \textbf{Dinâmica:} Descrever a dinâmica do segundo momento.

        \vspace{50pt}
        \noindent \emph{3º Momento:} Título do terceiro momento.
        \par\noindent\rule{.3\textwidth}{.5pt}
        \par\noindent \textbf{Dinâmica:} Descrever a dinâmica do terceiro momento.      
            
        \begin{align}
            x^2&=ax-\nabla\vec{F}_\mu\nu
        \end{align}
        
        \par\noindent \textbf{Avaliação:} Exemplo de questionário \cite{CARVALHO:2012}.        
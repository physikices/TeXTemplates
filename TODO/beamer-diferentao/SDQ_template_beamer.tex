%% LaTeX-Beamer template for KIT design
%% by Erik Burger, Christian Hammer
%% title picture by Klaus Krogmann
%%
%% version 2.1
%%
%% mostly compatible to KIT corporate design v2.0
%% http://intranet.kit.edu/gestaltungsrichtlinien.php
%%
%% Problems, bugs and comments to
%% burger@kit.edu

\documentclass[18pt]{beamer}

%% SLIDE FORMAT

% use 'beamerthemekit' for standard 4:3 ratio
% for widescreen slides (16:9), use 'beamerthemekitwide'
% for widescreen slide without sidebar use 'beamerthemekitwidenosidebar'

\usepackage{templates/beamerthemekitwide}
%\usepackage{templates/beamerthemekit}
%\usepackage{templates/beamerthemekitwidenosidebar}

% use this to disable the latex beamer navigation symbols
%\beamertemplatenavigationsymbolsempty

%% TITLE PICTURE

% if a custom picture is to be used on the title page, copy it into the 'logos'
% directory, in the line below, replace 'mypicture' with the 
% filename (without extension) and uncomment the following line
% (picture proportions: 63 : 20 for standard, 169 : 40 for wide
% *.jpg/*.png/*.pdf if you use pdflatex

%\titleimage{mypicture}

%% TITLE LOGO

% for a custom logo on the front page, copy your file into the 'logos'
% directory, insert the filename in the line below and uncomment it

%\titlelogo{mylogo}

% *.jpg/*.png/*.pdf if you use pdflatex

%% TikZ INTEGRATION

% the presentation starts here

\title[Short title]{Projeto Didático:\\ Forno de Indução}
\subtitle{Uma abordagem didática baseada na literatura}
\author{Rodrigo Nascimento}

\institute{Instrumentação para o Ensino de Física II}

% Bibliography

\usepackage[citestyle=authoryear,bibstyle=numeric,hyperref,backend=biber]{biblatex}
\addbibresource{templates/example.bib}
\bibhang1em

%% Inserção de codigos pessoais
\usefonttheme[onlymath]{serif}
%\usepackage[dvipsnames]{xcolor}

\begin{document}
% change the following line to "ngerman" for German style date and logos
	\selectlanguage{brazilian}

	%title page
	\begin{frame}
		\titlepage
	\end{frame}

%table of contents
	\begin{frame}{Sumário}
		\tableofcontents
	\end{frame}

	\section{Section 1}
	\subsection{Subsection 1.1}
	\begin{frame}{Example slide A}
		\begin{itemize}
			\item PCM, Citation: \cite{becker2008a} %\language
			\pause
			\item Bullet point 2
			\item \dots
		\end{itemize}
	\end{frame}

	\subsection{Subsection 1.2}
	\begin{frame}{Example slide B}
		\begin{block}{Block 1}
			\begin{itemize}
				\item Bullet point 1
				\pause
				\item Bullet point 2
				\item \dots
			\end{itemize}
		\end{block}
	\end{frame}

	\section{Section 2}
	\begin{frame}{Example slide C}
		\begin{exampleblock}{Example 1}
			\begin{itemize}
				\item Bullet point 1
				\pause
				\item Bullet point 2
				\item \dots
			\end{itemize}
		\end{exampleblock}
	\end{frame}

	\begin{frame}{Example slide D}
		\begin{alertblock}{Alert 1}
			\begin{itemize}
				\item Bullet point 1
				\pause
				\item Bullet point 2
				\item \dots
			\end{itemize}
		\end{alertblock}
	\end{frame}

	\begin{frame}{Dilatação $\Delta L$, $\Delta A$ e $\Delta V$}
		$$\Delta V=\gamma V_i\Delta T$$
		
		\tikzstyle{every picture}+=[remember picture]
		\tikzstyle{na} = [baseline=-.5ex]
		\begin{itemize}[<+-| alert@+>]
			\item Dif. de volumes -- ${\color{kit-red70}\Delta V}$
			\tikz[na] \node[coordinate] (n1) {};
		\end{itemize}
		\begin{equation*}
			\tikz[baseline]{
				\node[
					fill=kit-red70,
					anchor=base
				] (t1)
				{$ V_f-V_i$};
			}=				
			\tikz[baseline]{
				\node[
					fill=blue,
					ellipse,
					anchor=base
				] (t2)
				{$\gamma$};
			}
			\tikz[baseline]{
				\node[
					fill=orange,
					ellipse=base
				] (t3)
				{$V_i$};
			}
			\tikz[baseline]{
				\node[
					fill=green,
					anchor=base
				] (t4)
				{$\left(T_f-T_i\right)$};
			}
		\end{equation*}
		\begin{itemize}[<+-| alert@+>]
			\item Propriedade do material -- ${\color{blue}\gamma}$
			\tikz[na]\node [coordinate] (n2) {};
			\item Volume inicial -- ${\color{orange}V_i}$
			\tikz[na]\node [coordinate] (n3) {};
			\item Diferença de temperaturas -- ${\color{green}\Delta T}$
			\tikz[na]\node [coordinate] (n4) {};
		\end{itemize}
		
		\begin{tikzpicture}[overlay]
			\path[->]<1-> (n1) edge [bend left] (t1);
			\path[->]<2-> (n2) edge [bend right] (t2);
			\path[->]<3-> (n3) edge [out=0, in=-90] (t3);
			\path[->]<4-> (n4) edge [out=0, in=-90] (t4);
		\end{tikzpicture}
		\uncover<5>{$${\color{red}V_f}={\color{orange}V_i}\left[1+{\color{blue}\gamma}\left({\color{green}T_f}-{\color{green}T_i}\right)\right]$$}
	\end{frame}

	\appendix
	\beginbackup

	\begin{frame}[allowframebreaks]{References}
		\printbibliography
	\end{frame}

	\backupend

\end{document}

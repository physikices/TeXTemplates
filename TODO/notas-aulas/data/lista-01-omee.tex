\chapter{Módulo -- I}
% Campo para OBERVAÇÕES
\noindent \textbf{Resumo:} Lista desenvolvida com base no livro: \emph{Introdução à Física Estatística} do autor \citeauthor{SALINAS:2001}.

\par\noindent \textbf{Palavras chave:} \firstkey ; \secondkey ; \thirdkey ; \fourthkey.

\section{Introdução aos Métodos Estatísticos}
Revisão de Probabilidade e Estatística.
\subsection{Probabilidade}
\begin{definition}[Espaço Amostral $\Omega$]
	Conjunto de todosos resultados possíveis de um experimento/fenômeno aleatório.
\end{definition}
\begin{exmp}
	Altura dos alunos da turma ($A$): $\Omega_A=\{ x\in \mathbb{R} \vert 1,40\m\leq x\leq 2,10\m\}$.	
\end{exmp}
\begin{definition}[Evento]
	Qualquer subconjunto do Espaço Amostral, é chamdo de evento.
\end{definition}
\begin{exmp}
	$B$ é o conjunto de números primo de um dado de seis lados $B=\{1,2,3,5\}$.
\end{exmp}
\subsubsection{Operações}
\begin{itemize}
	\item União $\cup$ -- A união entre dois eventos $A$ e $B$, é um novo evento denotado por $A\cup B$, formado pelos elementos pertencentes a ambos os conjuntos.
	\item Intersecção $\cap$ -- A intersecção entre dois eventos $A$ e $B$, é um novo evento denotado por $A\cap B$, formado pelos elementos pertencentes simultâneamente a $A$ e a $B$.
	\item Complementar $A^C$ -- Todos os elementos do espaço amostral, que não pertencem ao evento $A$.
	\item Dois eventos $A$ e $B$, são ditos \textbf{mutuamente exclusivos} ou disjuntos se $A\cap B=\emptyset$
	\item Dois eventos $A$ e $B$, são ditos \textbf{complementares} se $A\cup B=\Omega$
\end{itemize}

\begin{definition}[Clássica]
	Suponha que um evento $A$ possa ocorrer de $k$ maneiras distintas num total de $n$ maneiras possíveis e igualmente prováveis. Então a \emph{probabilidade} de ocorrência do evento $A$ é $k/n$, definida como a \textbf{frequência relativa} do evento $A$.
\end{definition}
\begin{definition}[Moderna axiomática]
	Uma função $P(A)$ é denominada \emph{probabilidade} de ocorrência do evento $A$, se satisfaz as seguintes condições:
	\begin{enumerate}[label=\roman *)]
		\item $0\leq P(A)\leq 1$, $\forall$ $A\subset \Omega$;
		\item $P(\Omega)=1$;
		\item $P(\cup_{j=1}^nA_j)=\sum_{j=1}^n P(A_j)$, quando os elementos $A_j$ são disjuntos.
	\end{enumerate}
\end{definition}

\subsubsection{Propriedades}
\noindent Probabilidade de ocorrência do evento $A$ \textbf{ou} $B$
\begin{align}
	P(A\cup B)=P(A)+P(B)-P(A\cap B)
\end{align}
Probabilidade Condicional: Para dois eventos $A$ e $B$, com $P(B)>0$, a probabilidade de $A$ ocorrer dado que o evento $B$ já ocorreu é dada por
\begin{align}
	P(A|B)&=\frac{P(A\cap B)}{P(B)}
\end{align} 
\begin{exmp}
	No lançamento aleatório de um dado de seis lados $\Omega=\{1,2,3,4,5,6\}$, tem-se os seguintes eventos:
	\begin{itemize}
		\item $A:$ face superior é um número par $A=\{2,4,6\}$;
		\item $B:$ face superior é um número primo $B=\{1,2,3,5\}$;
		\item $C:$ face superior é um múltiplo de três $C=\{3,6\}$. 
	\end{itemize}
	Determine:
	\begin{enumerate}[label=\alph *)]
		\item $P(A\cup B)$
			\begin{eqnarray*}
				P(A)&=&\frac{1}{6}+\frac{1}{6}+\frac{1}{6}=\frac{1}{2}\\
				P(B)&=&\frac{1}{6}+\frac{1}{6}+\frac{1}{6}+\frac{1}{6}=\frac{2}{3}\\
				P(A\cap B)&=&\frac{1}{6}\\
				P(A\cup B)&=&\frac{1}{2}+\frac{2}{3}-\frac{1}{6}=1\qed
			\end{eqnarray*}
	\end{enumerate}
\end{exmp}
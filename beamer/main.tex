\documentclass[aspectratio=169]{beamer}
\newcommand{\autor}{Rodrigo Nascimento}

% Pacote de estilo da UDESC
\usepackage{style/udesc}

\graphicspath{{./img/}}

% --------------------------------------------- %
\input{modules/lstset.tex}
% maths
% --------------------------------------------- %
\usepackage{mathtools}
\usepackage{amsmath}
\usepackage{amssymb}
% --------------------------------------------- %

% phyiscs
% --------------------------------------------- %
\usepackage{physics}
% --------------------------------------------- %

% images
% --------------------------------------------- %
\usepackage{graphicx}
\usepackage{svg}
% --------------------------------------------- %

% others
% --------------------------------------------- %
\usepackage{hyperref}
\usepackage{lipsum}
% --------------------------------------------- %

\setbeamertemplate{section in toc shaded}[default][10]
\setbeamercolor{section in toc}{fg=secondary}
\setbeamercolor{subsection in toc}{fg=tertiary}

\setbeamertemplate{itemize items}[square]
\setbeamertemplate{frametitle continuation}{}

% --------------------------------------------- %

\usepackage[
  abnt-emphasize=bf,
  abnt-and-type=e,
  abnt-etal-text=it,
  num,
  overcite
]{abntex2cite}%Citações ABNT
\citebrackets []


% --------------------------------------------- %
% Início do documento
% --------------------------------------------- %
\begin{document}
%% 
% --------------------------------------------- %
\titulo{Física de Partículas}
\subtitulo{Cromodinâmica Quântica}
\newcommand{\github}{github.com/physikices}
\newcommand{\email}{rodrigo.nascimento@edu.udesc.br}
\newcommand{\website}{}
\frase{FMO2001}
\universidade{Licenciatura em Física -- 2023/2}
% --------------------------------------------- %
\capa
% \logo{
% 	\begin{tikzpicture}[overlay, remember picture]
% 		\node[left=0.2cm] at (current page.27){
% 				\includegraphics[height=.5cm]{assets/icons/sic.png}
% 			};
% 	\end{tikzpicture}
% }
% --------------------------------------------- %
% sumário
% --------------------------------------------- %
\AtBeginSection[]{
	\begin{frame}<beamer>
		\frametitle{Sumário\divisao}
		\tableofcontents[hideothersubsections]
	\end{frame}
}

\section{Seção I}
\subsection{Frame Block}
\subsection{Subseção I}
\subsection{Subseção II}

\section{Seção II}
\subsection{Subseção I}
\subsection{Subseção II}

\section{Seção III}
\subsection{Subseção I}
\subsection{Subseção II}
% --------------------------------------------- %
% slides
% --------------------------------------------- %
\secframe{Seção I}{Subtítulo de seção}
\begin{slide}{Seção I}{Frame Block}
	\begin{columns}
		\column{0.5\textwidth}
		\begin{block}{Normal Block}
			\begin{itemize}[<+(1)->]
				\item item 1
				\item item 2
			\end{itemize}
		\end{block}
		\begin{alertblock}{Alert Block}
			\begin{itemize}[<+(1)->]
				\item  item 1
				\item  item 2
			\end{itemize}
		\end{alertblock}

		\column{0.5\textwidth}
		\begin{exampleblock}{test}
			\imagem{.5}{bncc_just-01}{Fonte \cite{CATARINA:2021}}
		\end{exampleblock}
	\end{columns}
\end{slide}

\secframe{Seção II}{Subtítulo de seção}
\begin{slide}{Seção II}{Equações}
	\centering
	\begin{tikzpicture}
		\node[root]{$R_{\mu\nu}-\frac{1}{2}g_{\mu\nu}R=T_{\mu\nu}$};
	\end{tikzpicture}

	\begin{exampleblock}{}
		\begin{align*}
			\begin{split}
				\vec{\nabla}\cdot\vec{D} &= \rho_{f} \\
				\vec{\nabla}\cdot\vec{B} &= 0	\\
				\vec{\nabla}\times \vec{E} &= -\partial_t \vec{B}\\
				\vec{\nabla}\times \vec{H} &= \vec{J}_{f}+\partial_t \vec{D}
			\end{split}
		\end{align*}	
	\end{exampleblock}
\end{slide}

\begin{frame}[plain]
	\begin{tikzpicture}
		\node[basic](a1){Teste};	
		\node[root, right=.5cm of a1](a2){Teste};
		\node[level 1, right=.5cm of a2](a3){Teste};
		\node[level 2, right=.5cm of a3](a4){Teste};
		\node[level 3, right=.5cm of a4](a5){Teste};
	\end{tikzpicture}	
\end{frame}

% --------------------------------------------- %
\begin{frame}[allowframebreaks]
	\frametitle{Referências\divisao}
	\bibliography{referencias.bib}
\end{frame}


\contato{%
	Contato: \\
	\autor{} \\
	\email{} \\
	\github{} \\
	\website{}
}

\capadetras{}

\end{document}

\documentclass[xcolor=dvipsnames]{beamer}

\usepackage[brazil]{babel}
\usepackage[utf8]{inputenc}
\usefonttheme[onlymath]{serif}

\usetheme{Madrid}
\usepackage{beamercolorthemestd}
\useoutertheme{miniframes} %Alternatively: miniframes, infolines, split
\useinnertheme{circles}
\pgfdeclareimage[height=0.5cm]{logo}{assets/logo.png}
\logo{\pgfuseimage{logo}}

\title{título}
\subtitle{subtítulo}
\author[disciplina]{autor}
\date{\today}
\institute[UDESC]{Universidade do Estado de Santa Catarina\\Centro de Ciências Tecnológicas\\Campus Joinville}

\begin{document}

\begin{frame}
  \titlepage
\end{frame}

\begin{frame}{Sumário}
  \tableofcontents[pausesections]
\end{frame}

% Presentation structure

\begin{frame}{Exiting results}
  \section{Existing results}
  \subsection{Method 1}
  \subsection{Method 2}
  \subsection{Method 3}
  \framesubtitle{Teste}
  \begin{block}{Open Questions}
	 Is every even number the sum of two primes?
	 \cite{Goldbach1742}
  \end{block}
\end{frame}


\begin{frame}{Comparative Study}
  \section{Comparative study}
  \framesubtitle{Teste}
  \begin{columns}
	 \column{.4\textwidth}
	 \begin{block}{Answered Questions}
		How many primes are there?
	 \end{block}
	 \column{.4\textwidth}
	 \begin{block}{Open Questions}
		23
		Is every even number the sum of two primes?
	 \end{block}
  \end{columns}
\end{frame}

\begin{frame}[fragile]
  \frametitle{An Algorithm For Finding Primes Numbers.}
  \begin{semiverbatim}
		\uncover<1->{\alert<0>{int main (void)}}
		\uncover<1->{\alert<0>{\{}}
		\uncover<1->{\alert<1>{ \alert<4>{std::}vector<bool> is_prime (100, true);}}
		\uncover<1->{\alert<1>{ for (int i = 2; i < 100; i++)}}
		\uncover<2->{\alert<2>{ if (is_prime[i])}}
		\uncover<2->{\alert<0>{ \{}}
		\uncover<3->{\alert<3>{ \alert<4>{std::}cout << i << " ";}}
		\uncover<3->{\alert<3>{ for (int j = i; j < 100;}}
		\uncover<3->{\alert<3>{ is_prime [j] = false, j+=i);}}
		\uncover<2->{\alert<0>{ \}}}
		\uncover<1->{\alert<0>{ return 0;}}
		\uncover<1->{\alert<0>{\}}}
  \end{semiverbatim}
  \visible<4->{Note the use of \alert{\texttt{std::}}.}
\end{frame}

\begin{frame}{Referências}
  \section{Referencias}
  \begin{thebibliography}{10}
	 \bibitem{Goldbach1742}[Goldbach, 1742]
	 Christian Goldbach.
	 \newblock A problem we should try to solve before the ISPN ’43 deadline,
	 \newblock \emph{Letter to Leonhard Euler}, 1742.
\end{thebibliography}
\end{frame}

\end{document}

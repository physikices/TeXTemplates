\documentclass[aspectratio=169]{beamer}

% Pacote de estilo da UDESC
\usepackage{style/udesc}
\usepackage{listings}
\usepackage{hyperref}
\setbeamertemplate{itemize items}[square]
\usepackage[abnt-emphasize=bf,abnt-and-type=e,alf]{abntex2cite}%Citações ABNT

% Incluir arquivos da pasta figuras
\graphicspath{{./img/}}
\setbeamertemplate{frametitle continuation}{}

% aPacote de texto aleatório
\usepackage{lipsum}
\lstset{ 
	basicstyle=\footnotesize,        % the size of the fonts that are used for the code
	breakatwhitespace=false,         % sets if automatic breaks should only happen at whitespace
	breaklines=true,                 % sets automatic line breaking
	captionpos=b,                    % sets the caption-position to bottom
	deletekeywords={...},            % if you want to delete keywords from the given language
	escapeinside={\%*}{*)},          % if you want to add LaTeX within your code
	extendedchars=true,              % lets you use non-ASCII characters; for 8-bits encodings only, does not work with UTF-8
	firstnumber=0,                   % start line enumeration with line 1000
	frame=single,	                   % adds a frame around the code
	keepspaces=true,                 % keeps spaces in text, useful for keeping indentation of code (possibly needs columns=flexible)
	language=Java,                   % the language of the code
	morekeywords={*,...},            % if you want to add more keywords to the set
	numbers=none,                    % where to put the line-numbers; possible values are (none, left, right)
	numbersep=0pt,                   % how far the line-numbers are from the code
	rulecolor=\color{black},         % if not set, the frame-color may be changed on line-breaks within not-black text (e.g. comments (green here))
	showspaces=false,                % show spaces everywhere adding particular underscores; it overrides 'showstringspaces'
	showstringspaces=false,          % underline spaces within strings only
	showtabs=false,                  % show tabs within strings adding particular underscores
	stepnumber=2,                    % the step between two line-numbers. If it's 1, each line will be numbered
	tabsize=1,	                     % sets default tabsize to 2 
	basicstyle=\fontsize{7}{8}\selectfont\ttstyle,
	keywordstyle=\color{udescred},
	commentstyle=\fontsize{7}{8}\selectfont\ttstyle\color{gray65udesc},
	stringstyle=\color{orange},
}

% Início do documento
\begin{document}
%%
%%	Incluir \capa para os slides
%% 
\titulo{Título da Apresentação}
\subtitulo{Subtítulo da Apresentação}
\newcommand{\autor}{Nome do Palestrante}
\newcommand{\github}{github.com/physikices}
\newcommand{\email}{rodrigo.nascimento@edu.udesc.br}
\newcommand{\website}{}
\frase{Seminário -- SIGLA2023}
\universidade{credenciais}
\capa
\logo{
	\begin{tikzpicture}[overlay, remember picture]
		\node[left=0.2cm] at (current page.27){
		\includegraphics[height=.5cm]{style/sic.png}
	};
	\end{tikzpicture}
}


\AtBeginSection[]{
	\begin{frame}<beamer>
		\frametitle{Sumário\hrule}
		\tableofcontents[currentsection]
  \end{frame}}

\section{Seção I}
\subsection{Frame Blocks}
\subsection{Subseção I}
\subsection{Subseção II}

\section{Seção II}
\subsection{Subseção I}
\subsection{Subseção II}

\section{Seção III}
\subsection{Subseção I}
\subsection{Subseção II}

\secframe{Seção I}{Subtítulo de seção}
\begin{slide}{Seção I}{Frame Blocks}
	\begin{columns}
		\column{0.5\textwidth}
		\begin{block}{Normal Block}
			\begin{itemize}[<+(1)->]
					\item item 1
					\item item 2
			\end{itemize}
		\end{block}
		\begin{exampleblock}{Example Block}
			\begin{itemize}[<+(1)->]
				\item  item 1
				\item  item 2
			\end{itemize}
		\end{exampleblock}
		\begin{alertblock}{Alert Block}
			\begin{itemize}[<+(1)->]
				\item  item 1
				\item  item 2
			\end{itemize}
		\end{alertblock}

		\column{0.5\textwidth}
		\begin{exampleblock}{}
			\imagem{.8}{bncc_just-01}{Fonte \cite{CATARINA:2021}}
		\end{exampleblock}
	\end{columns}
\end{slide}

\secframe{Seção II}{Subtítulo de seção}
\begin{slide}{Seção II}{Equações}
	\centering
	\begin{tikzpicture}
		\node[root]{$R_{\mu\nu}-\frac{1}{2}g_{\mu\nu}R=T_{\mu\nu}$};
	\end{tikzpicture}

	\begin{exampleblock}{}
		\begin{align*}
			\begin{split}
				\vec{\nabla}\cdot\vec{D} &= \rho_{f} \\
				\vec{\nabla}\cdot\vec{B} &= 0	\\
				\vec{\nabla}\times \vec{E} &= -\partial_t \vec{B}\\
				\vec{\nabla}\times \vec{H} &= \vec{J}_{f}+\partial_t \vec{D}
			\end{split}
		\end{align*}	
	\end{exampleblock}
\end{slide}

\begin{frame}[allowframebreaks]
	\frametitle{Referências\hrule}
	% \framesubtitle{\hrule}
	\bibliography{referencias.bib}
\end{frame}

\contato{%
	Contato: \\
	\autor{} \\
	\email{} \\
	\github{} \\
	\website{}
}
\capadetras{}

\end{document}

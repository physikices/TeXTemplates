%--------------| Q01 |--------------------
\addcontentsline{toc}{section}{Problema 01}
\begin{prob}
  Para o ensemble grande canônico
  \begin{enumerate}[label=\alph *)]
    \item Derive a função de partição.
    \item Obtenha a conexão com termodinâmica.
  \end{enumerate}
  \begin{sol}
    No ensemble grande canônico, o sistema $S$ é posto em contato com um ambiente de temperatura e número de partículas fixos, de forma que $S$ pode trocar energia e partículas com este ambiente, mantendo sempre o volume fixo. Os microestados acessíveis ao sistema com esta restrição, será denotado por $j$, de maneira que os valores médios de energia $U$ e número de partículas $N$ é simplesmente a probabilidade $P_{j}$ de ocorrência destes estados
    \begin{align}
      N&=\sum_{j}N_jP_{j},\\
      U&=\sum_{j}E_{j}P_{j}
    \end{align}
    Na representação do grande potencial termodinâmico $\Phi =U-TS-\mu N$, com a entropia na forma dada por Shannon $S=-k_B\sum_jP_j\ln P_{j}$, a interpretação microscópica do grande potencial termodinâmico, será a soma sobre todos os microestados do grande ensemble canônico
    \begin{align}
      \Phi&=\sum_j\left(E_j+k_BT\ln P_j-\mu N_j\right)P_j
    \end{align}
    É necessário encontrar qual a distribuição de probabilidades que minimiza o grande pontencial termodinâmico. Constuíndo uma função de lagrange para $\Phi$ e impondo a condição de que $P_j$ seja normalizável
    \begin{dmath*}
      \mathcal{L}=\Phi+\lambda\left[\sum_jP_j-1\right]\condition{desde que $\displaystyle\parder{\mathcal{L}}{\lambda}=0$}
    \end{dmath*}
    e
    \begin{align}
      \parder{\mathcal{L}}{P_j}&=E_j+k_BT\left(1+\ln P_j\right)-\mu N_j+\lambda=0
    \end{align}
    Resolvendo pra $P_j$
    \begin{align}
      \begin{split}
        \ln P_j&=-1-\lambda\beta-\beta\left(E_j-\mu N_j\right)\\
        P_j&=\E^{-1-\lambda\beta}\exp\left[-\beta\left(E_j-\mu N_j\right)\right]
      \end{split}
    \end{align}
    Da condição de normalização imposta decorre diretamente que
    \begin{align}
      \sum_JP_j=1\Longrightarrow \E^{-1-\lambda\beta}=\frac{1}{\displaystyle\sum_j\exp\left[-\beta\left(E_j-\mu N_j\right)\right]}
    \end{align}
    escolhendo a constante de normalização convenientemente como sendo $1/\Xi$ obtemos
    \begin{align}
      P_j&=\frac{1}{\Xi}\exp\left[-\beta\left(E_j-\mu N_j\right)\right]
    \end{align}
    onde a função de partição do ensemble grande canônico é dada por
    \begin{align}
      \label{eq:fparticao-gde-canonico}
      \addtolength{\fboxsep}{5pt}
      \boxed{
        \begin{gathered}
          \Xi=\sum_j\exp\left[-\beta\left(E_j-\mu N_j\right)\right]
        \end{gathered}
      }      
    \end{align}
    Para fazer a conexão com a termodinâmica, vamos reescrever a \eqref{eq:fparticao-gde-canonico} fatorizando o termo relacionado ao número de partículas
    \begin{align}
      \Xi&=\sum_N\E^{\beta \mu N}\sum_{j_n}\E^{-\beta E_{j_n}}
    \end{align}
    mas
    \begin{align}
      \sum_{j_n}\E^{-\beta E_{j_n}}&=Z_j
    \end{align}
    em que $Z_j$ é a função de partição canônica de maneira que
    \begin{align}
      \Xi&=\sum_N\E^{\beta \mu N}Z_N
    \end{align}
    Introduzindo o termo $Z_N$ na exponêncial e substituindo o somatório pelo seu termo máximo, encontramos
    \begin{align}
      \begin{split}
        \Xi&=\sum_N\exp\left[\beta \mu N+\ln Z_N\right]\sim \exp\left[-\beta\min_N \left(-k_BT\ln Z-\mu N\right)\right]
      \end{split}
    \end{align}
    identificamos a energia livre de Helmholtz $F=-k_BT\ln Z$ e notando que
    \begin{align}
      \begin{split}
        F-\mu N&=U-TS-\mu N\\
        F-\mu N&=\Phi
      \end{split}
    \end{align}
    isto é
    \begin{align}
      \Xi\longrightarrow \E^{-\beta\Phi}
    \end{align}
  \end{sol}
  e para um fluído puro
  \begin{align}
    \Phi=\Phi(T,V,\mu)\to -\frac{1}{\beta}\ln\Xi(T,V,\mu)
  \end{align}
  a conexão com a termodinâmica se da por
  \begin{align}
    \addtolength{\fboxsep}{5pt}
    \boxed{
      \begin{gathered}
        \phi(T,\mu)=-\frac{1}{\beta}\lim_{V\to\infty}\frac{1}{V}\ln\Xi(T,V,\mu)
      \end{gathered}
    }
  \end{align}
\end{prob}
